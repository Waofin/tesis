\documentclass[spanish]{article}
\usepackage[T1]{fontenc}
\usepackage[spanish]{babel}
\selectlanguage{spanish}
\usepackage{babel}
\usepackage{graphicx}
\usepackage{float}
\usepackage[margin=2cm]{geometry}
\usepackage[utf8]{inputenc}
\usepackage[table,xcdraw]{xcolor}
\usepackage{tabularx}


% Para eliminar el título "references" de la bibliografía
\usepackage{etoolbox}
\patchcmd{\thebibliography}{\section*{\refname}}{}{}{}
\begin{document}

\title{
    \includegraphics[height=2cm]{figures/UBB.png}\\
    UNIVERSIDAD DEL BÍO-BÍO \\
    FACULTAD DE CIENCIAS EMPRESARIALES\\
    Escuelas de: Ingeniería Civil Informática \\
    Ingeniería En Ejecución En Computación E Informática\\
    \textbf{Propuesta de Proyecto de Títulación} }
\maketitle
\section{Identificación del/los estudiante(s)}

% --- Comentarios profe previo entrega 1 final ---
% Detalles referencias corregidos ---
% NO dare mas detalles de contenido de la propuesta porque tu profesor guia debe revisar que este de acuerdo a lo que acordaron.
% No olvidar que debes hacer el calendario anual y dividirlo por anteproyecto y proyecto de titulo


\subsection{Estudiante 1 }
\begin{itemize}
    \item NOMBRE: XXXX XXXX XXX XXX
    \item DIRECCIÓN: XXXX 545, XXXx, Concepción
    \item TELÉFONO: +569xxxxxxxx
    \item E-MAIL: xxxxxx@alumnos.ubiobio.cl
    \item CARRERA: Ingeniería Civil en Informática
    \item DEPTO.: Sistemas de Información 
\end{itemize}
**Repetir aquí los datos anteriores si hay más de un alumno, sino borrar esta linea.


\section{Identificación de Guía(s), asesores y supervisión}
\subsection{Profesor Guía}
\begin{itemize}
    \item NOMBRE : xxxx
    \item E-MAIL : xxxx@ubiobio.cl    ~~~~~~~~~~~~~~~~~~~~~    Firma: 
\end{itemize}


\subsection{Personas, Instituciones o Empresas en que se solicitará apoyo y/o asesoría}
**si corresponde, sino eliminar
\begin{itemize}
    \item NOMBRE : xxxx
    \item E-MAIL : xxxx@ubiobio.cl
    \item Rubro  : xxxxxxxx    ~~~~~~~~~~~~~~~~~~~~~~~~~~~    Firma: 

\end{itemize}

\subsection{Responsable de la empresa que supervisará al alumno}
**si corresponde, sino eliminar
\begin{itemize}
    \item NOMBRE : xxxx
    \item E-MAIL : xxxx@ubiobio.cl
    \item Rubro  : xxxxxxxx    ~~~~~~~~~~~~~~~~~~~~~~~~~~~    Firma: 

\end{itemize}

\section {Proyecto de Título}

\subsection{Título que identificará la actividad de Titulación}
**la idea es buscar un nombre que represente el sistema y no solo un nombre de fantasía. Debe tener estrecha relación con el objetivo, es lo último que se escribe.

"xxx xxxxxxx xxxx xxxxx"
\subsection{Modalidad de la propuesta}
\begin{center}
    \begin{tabular}{|c|l|}
    \hline
    & Desarrollo de Software \\
    \hline
    &  Investigación con desarrollo de Software\\
    \hline
\end{tabular}
\end{center}

\subsection{Presentación del área y su problemática}

Aqui va introducción a la sección, por ejemplo indicar que se explicara y como esta dividida.

\subsubsection{Presentación de la Empresa}\label{empresa}
*Origen, ubicación, a que se dedica.
*Contexto organizacional que ayude a comprender el área en la cual se centra el proyecto

\subsubsection{Presentación de los procesos del ámbito del proyecto}\label{procesos}

*Descripción de los pasos y actores relacionados en los procesos involucrados en el área de interés… como se hacen las cosas HOY EN DIA.

\subsubsection{Descripción del Problema}\label{problema}
* Analizar si los procesos se pueden mejorar, que errores o problemas existen.
** De la descripción de esta situación actual (\ref{procesos}) se identifica y especifica el problema que ha motivado la necesidad del sistema, lo cual definirá el objetivo del sistema. Si lo anterior no es coherente significa, simplemente, que el proyecto ha sido mal conceptualizado

\subsection{Análisis de los Principales Trabajos Realizados en el área o tema de la propuesta}\label{arte}
** Si los hay, describir en qué se diferencian, cuál es el aporte de este proyecto

EJEMPLO1
\begin{enumerate}
    \item Un análisis comparativo de los sistemas de recomendación basado en opiniones de aspectos de ítems extraídas de las opiniones de usuarios \cite{gutierrez2008estudio}.
    
    En este trabajo se analiza la extracción de opiniones en inglés, en base al estado-del-arte, logrando presentar mejoras con el enfoque propuesto. En este Proyecto de Título, se propone analizar esto, e, implementar los algoritmos propuestos con la capacidad de analizar opiniones en español.
    
    \item Xxxxxxxx xxxxxxx xxxxxx xxx  
    
    En este trabajo también Xxxxxxxx xxxxx xxxxxxxxx xxxxxxxxxx  
    
    
\end{enumerate}

EJEMPLO 2
\begin{itemize}
    \item Se encuentran en la biblioteca de la Universidad los siguientes trabajos …...
    \item No se encuentra trabajo previo en la Universidad del Bío-Bío, pero sí en otras universidades de Latinoamérica.
    \begin{enumerate}
        \item Video Game ....
        \item Aprendizado .....
    \end{enumerate}
    \item También, hay trabajos publicados en revistas de investigación que se relacionan con el trabajo a realizar, estos serían los siguientes:
    \begin{enumerate}
        \item En el artículo ....
        \item En el artículo …
    \end{enumerate}
    \item Libros o aplicaciones encontradas
\end{itemize}


EJEMPLO 3

Se investigó en el buscador ------------ con fecha -------------- del año 202x con el propósito de conocer que otras soluciones existían actualmente y en qué se diferenciaban del software que se propone en este proyecto.
Por cada una describir funciones, características, costos, valoración de usuarios

\begin{enumerate}
    \item Xxxxxxxxxxxxx xxxxxxxxxxxxxxxxxxx xxxxxxxxxxxxxxxxxx xxxxxxxxxxxxxxxxxxxxx
    
    \item Xxxxxxxxxxxxx xxxxxxxxxxxxxxxxxxx xxxxxxxxxxxxxxxxxx xxxxxxxxxxxxxxxxxxxxx
\end{enumerate}



\subsection{Propuesta}

** Qué solución software usted propone?(SW) o 

** ¿Qué se espera de la investigación (INV)

**¿Porqué es necesario resolver este problema? ¿Qué beneficios se obtienen al resolverlo?

\subsection{Justificación del Proyecto}
* ¿Por qué la opción de su proyecto de tesis que Ud. propone es MEJOR que las soluciones existentes (revisadas en Sección \ref{arte}). Ser Mejor puede implicar ser mas económico (desarrollo, hosting, pago por uso), ser justo lo que necesitan, más de lo que necesitan, etc.

Ejemplo de citación de bibliografia en el texto:

El análisis de los aspectos de un ítem (criterios que puedan influir en la preferencia del ítem) ha sido probado como un método exitoso para mejorar la calidad de recomendaciones \cite{gutierrez2008estudio}, esto basado en la amplia existencia de aplicaciones que permiten expresar opiniones personales respecto a diferentes ítems (además de las típicas calificaciones de 1-5 estrellas), señalando textualmente su preferencia o rechazo. Estas opiniones pueden ser analizadas mediante algoritmos, para extraer información sobre aspectos del ítem que influyen en el gusto de los usuarios, y esa es la problemática que se desea abordar en este trabajo de Proyecto de \cite{gutierrez2008estudio} Título  \cite{gutierrez2008estudio, gutierrez2008estudio,gutierrez2008estudio}.

Toda Figura debe ser descrita en texto, no solo indicar que se ve en Figura X.


Inicio parrafo final: Por todo lo anterior, se propone en este Proyecto de Título Xxxxxxx xxxxxx xxxxxxxx xxxxx xxxxxx xxxxxx xxxxxxx xxxxx xxxxxxxxx xxxxxx xxxxxx


%%%%%%%%%%%%%%%%%%%%%%%%%%%%%%%%%%%%%%%%%%%%%%%%%%%%%%%%%%%%%%%%%%%%
\subsection{Objetivos}
Intro a la subsección
\subsubsection{Objetivo General}
**Qué problema se tratará de resolver en el proyecto de Titulación, se debe usar verbos Estudiar, identificar ...… (verbo)

Verbo xxxxxxxxxxxxxxxxxxx xxxxxxxxxxxxxxxxxx xxxxxxxxxxxxxxxxxxxxx


\subsubsection{Objetivos Específicos}
**Especificación de problemas a resolver asociados al proyecto, aquí están incluidos Estudios del Arte, Mercado, Análisis Variados, desarrollo de Sw, Evaluaciones, pruebas, otros. se debe usar verbos Estudiar, identificar ...… (verbo)
\label{obj}

Ejemplo 1

\begin{enumerate}
    \item Revisar articulos del área ... 
    \item Analizar técnicas xxxxxxxxxxxx
    \item IXxxxxxxx xxxxx xxxxxxxxxxx
    \item Xxxxxxxxxx xxx xxxx xxxxxxxxxxxxxx
\end{enumerate}

Ejemplo 2 

\begin{enumerate}
    \item Investigar tecnología ...
    \item Analizar
    \item Elaborar propuesta de solución basado en las necesidades y expectativas de los usuarios respecto a las tenencias de los equipos.
    \item Diseñar la solución software de arriendo de equipamiento computacional
    \item Implementar el software de arriendo de equipamiento computacional
\end{enumerate}



\subsubsection{Descripción de la actividades para lograr los objetivos Específicos}\label{actividades}
**Puede haber una actividad o más por objetivo específico, no utilizar verbos en las actividades, por ejemplo, investigación, análisis, estudio, revisión ...

Si el objetivo específico es \textbf{“Definir los requisitos funcionales y no funcionales de la solución software”} las actividades pueden ser:
\begin{enumerate}
    \item Revisión de aplicaciones móviles y web similares en Google y Store (Android y IOs)
    \item Revisión de proyectos relacionados desarrollados en la UBB, búsqueda en Werken
    \item Recopilación de necesidades de usuarios clave
    \item Comparación de soluciones similares
    \item Definición de propuesta preliminar
    \item Validación de Maqueta con usuarios
\end{enumerate}

\subsection {Descripción de los Aspectos Fundamentales de la Metodología a utilizar en el Desarrollo de Software o en Investigación}

\subsubsection*{Ejemplo SW}
La siguiente tabla caracteriza el proyecto a desarrollar, en distintos aspectos:

\begin{tabular}{|l|c|l|}
    \hline
    Ítem & Nivel & Descripción \\
    \hline
    Experiencia en el problema &  & conocemos o sabemos de qué se trata este tipo de problema, hemos \\  
    & & trabajadoen el rubro, hemos trabajado en procesos similares,  \\
    & & aunque de rubros distintos\\
    \hline
    Tamaño del problema &  & cantidad de procesos, procedimientos, manuales a revisar o cantidad \\
    & & de usuarios a los que debo consultar   \\

    \hline
    Complejidad del problema &  & lo que debo entender (procesos, procesamiento, cálculos, decisiones, \\
    & &  etc) es simple o poco claro, requiere dedicación.\\    

    \hline
    Tamaño del software &  & cantidad de funcionalidades del software que debo construir  \\

    \hline
    Complejidad del software &  & las funcionalidades que el software debe IMPLEMENTAR tienen \\
        & & cálculos complejos, con operaciones complejas, implementa con\\
        & & hilos/múltiples procesadores, o se requiere usar un hw/sw específico \\
    \hline
    Experiencia en el software &  & conocemos o tenemos experiencia desarrollando funcionalidades \\
    & & similares a las que este software requiere \\
    \hline
    Modularidad &  & la funcionalidad del sw pude ser dividida para implementarse en \\
    & &  paralelo, por partes, que se puedan ensamblar (Máxima cohesión \\
    & & mínimo acoplamiento). \\
    \hline
\end{tabular}

Considerando los factores relacionados con el problema, se concluye que el proyecto xxx se considera riesgoso, y que los factores relacionados a la solución técnica xxxx nivel de riesgo POR LO TANTO  se ha seleccionado la metodología XXXX YA QUE ….

\subsection{Plan de trabajo a desarrollar}

A**Si es anteproyecto, debe ser calendarización también para proyecto de título (2 semestres). En el caso que sea solo proyecto de título, calendarizar para un solo semestre

\subsection{Programación Temporal}

\begin{center}
\begin{tabular}{|c|c|c|c|c|}
\hline
\textbf{Etapa} & \textbf{Número actividad} & \textbf{Fecha inicio} & \textbf{Fecha final} & \textbf{Duración (días)} \\
\hline
Anteproyecto & Act N°1 & 01-05-2025 & 30-05-2025 & 30 \\ \hline
Anteproyecto & Act N°2 & 05-05-2025 & 25-05-2025 & 20 \\ \hline
PT        & Act N°3 & 16-06-2025 & 16-09-2025 & 90 \\ \hline
PT        & Act N°4 & 15-07-2025 & 30-09-2025 & 75 \\ \hline
PT        & Act N°5 & 01-10-2025 & 10-10-2025 & 10 \\
\hline
\end{tabular}
\end{center}

\section{Referencias}
* Utilizar Estilo IEEE para las referencias.

\bibliographystyle{ieeetr}
\bibliography{references.bib}

\newpage

\begin{center}
    \textbf{LA PRESENTE SOLICITUD DE INSCRIPCIÓN DE LA ACTIVIDAD DE TITULACIÓN SIGNIFICA UN COMPROMISO DE CUMPLIR LO ESTIPULADO EN ELLA.}
\end{center}

%%%%%%%%%%%%%%%%%%%%%%%%%%%%%%%%%%%%%1 alumno 
\begin{center}
    \begin{tabular}{ccc}
        & &\\
        & &\\
        & &\\
        & &\\
                & &\\
        & &\\
     & ~~~~~~~~~~~~~~~~~~~~~~~~~~~  &  \\

    Firma   &  &   \\

    Nombre alumno &  &  \\
\end{tabular}
\end{center}

%%%%%%%%%%%%%%%%%%%%%%%%%%%%%%%%%%%%% 2 alumno 

\begin{center}
    \begin{tabular}{ccc}
        & &\\
        & &\\
        & &\\
        & &\\
                & &\\
        & &\\
     & ~~~~~~~~~~~~~~~~~~~~~~~~~~~  &  \\

    Firma   &  & Firma  \\

    Nombre Alumno 1 &  & Nonmbre Alumno 2 \\
\end{tabular}
\end{center}

%%%%%%%%%%%%%%%%%%%%%%%%%%%%%%%%%%%%%
\subsection*{FECHA PRESENTACIÓN SOLICITUD: XX de Mes de 2025}

\subsection*{RESOLUCIÓN DIRECTOR DE PROGRAMA/JEFE DE CARRERA}

\begin{itemize}
    \item \textbf{APROBADO~~}: 
    \item \textbf{REPROBADO}: 
\end{itemize}
\subsection*{FECHA RESOLUCIÓN:}



\begin{center}
    \begin{tabular}{c}
        \\
        \\
                \\
        \\
        \\
        \\
        \\
       \\
        
        Firma \\
        
        \textbf{DIRECTOR DE PROGRAMA/JEFE DE CARRERA} \\
    \end{tabular}
\end{center}

\newpage
\appendix

 

\end{document}


