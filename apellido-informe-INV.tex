  \documentclass[final,letterpaper,oneside,authoryear,11pt,singlespace,spanish]{ezthesis}
\usepackage[spanish]{babel}
\addto\captionsspanish{
\def\bibname{Referencias}
\def\tablename{Tabla}
}
\usepackage{hyperref}
\hypersetup{
    colorlinks=true,
    linkcolor=blue,
    filecolor=blue,      
    urlcolor=blue,
    citecolor=blue
}
\usepackage{rotating}
%%\usepackage{algorithm}
%\usepackage{algpseudocode}
%\usepackage[noend]{algpseudocode}
\usepackage{siunitx}
\usepackage{mathtools}
\usepackage[T1]{fontenc}
\usepackage[utf8]{inputenc}
\usepackage{anysize}
\usepackage{txfonts}
\usepackage{wrapfig} %figura entre texto
\usepackage{times}
\usepackage{listings}
\usepackage{acronym}
\usepackage{array}
\usepackage{textcomp}
\usepackage{fancyvrb}
\usepackage{fancyhdr}
\usepackage{wallpaper}
\usepackage{verbatim}%Agregado por Luis
\usepackage{latexsym}
\usepackage{url}
\usepackage{multicol}
\usepackage{graphicx}
\usepackage{multirow}
\usepackage{float}
\usepackage{lmodern}
\usepackage{eso-pic}
\usepackage{subfig}
%\usepackage{amssymb}
\usepackage{color}
\usepackage{ragged2e}
\usepackage[all]{xy}
\usepackage{xspace,epic,eepic}
\usepackage{algorithmic}
\usepackage[ruled,vlined]{algorithm2e}
%\usepackage[ruled,vlined,linesnumbered,titlenotnumbered, %portuguese]{algorithm2e}
\usepackage{enumitem}
\usepackage[T1]{fontenc}
\usepackage[spanish]{babel}
\selectlanguage{spanish}
\usepackage{babel}
\usepackage{graphicx}
\usepackage{booktabs} 
\usepackage{float}
\usepackage{pgfgantt}
\usepackage{pdflscape} 
\usepackage{tikz}
\usepackage{amsmath}   % Para ecuaciones
\usepackage{amssymb}   % Símbolos matemáticos
% Entornos personalizados
\newcommand{\RV}{\text{RV}}    % Velocidad Radial
\newcommand{\Tr}{\text{Tr}}    % Tránsitos
\usetikzlibrary{shapes.geometric, arrows.meta, positioning}

\tikzstyle{block} = [rectangle, rounded corners, minimum width=3cm, minimum height=1cm,text centered, draw=black, fill=blue!20]
\tikzstyle{arrow} = [thick,->,>=stealth]

\usepackage[]{geometry}
\usepackage[table,xcdraw]{xcolor}
\usepackage[colorlinks=true, linkcolor=blue, urlcolor=blue]{hyperref}
\setlist{topsep=0pt,noitemsep}
\setcounter{tocdepth}{3}
\marginsize{3cm}{2.5cm}{2.5cm}{2.5cm}

%Definicion de Colores
\definecolor{gray97}{gray}{.97}
\definecolor{gray75}{gray}{.75}
\definecolor{gray45}{gray}{.45}
\definecolor{listinggray}{gray}{0.9}
\definecolor{lbcolor}{rgb}{0.9,0.9,0.9}
\definecolor{codegreen}{rgb}{0,0.6,0}
\definecolor{codegray}{rgb}{0.5,0.5,0.5}
\definecolor{codepurple}{rgb}{0.58,0,0.82}
\definecolor{backcolour}{rgb}{0.95,0.95,0.92}
\newcommand\rojo[1]{\textcolor[rgb]{1,0,0}{\textbf{#1}}}
\newcommand\red[1]{\textcolor[rgb]{1.00,0.00,0.00}{#1}}
\newcommand\azul[1]{\textcolor[rgb]{0,0,1}{\textbf{#1}}}
\newcommand\blue[1]{\textcolor[rgb]{0,0,1}{{#1}}}
\newcommand\verde[1]{\textcolor[rgb]{0,.5,0.2}{\textbf{#1}}}
\newcommand\naranjo[1]{\textcolor[rgb]{1.00,0.36,0.06}{\textbf{#1}}}
%\ifCLASSINFOpdf
%\else
%\fi
%\hyphenation{pa-la-bra}

\lstset{
	%backgroundcolor=\color{lbcolor},
	tabsize=4,
	rulecolor=,
	language=SQL,
        basicstyle=\small,
        upquote=true,
        aboveskip={1.5\baselineskip},
        columns=fixed,
        showstringspaces=false,
        extendedchars=true,
        breaklines=true,
        prebreak = \raisebox{0ex}[0ex][0ex]{\ensuremath{\hookleftarrow}},
        %frame=single,
        showtabs=false,
        showspaces=false,
        showstringspaces=false,
        identifierstyle=\ttfamily,
        keywordstyle=\color[rgb]{0,0,0},
        commentstyle=\color[rgb]{0,0,0},
        stringstyle=\color[rgb]{0,0,0},
}


%\renewcommand{\labelenumi}{\arabic{enumi}.} % (1., 2., 3.,...)
%\renewcommand{\labelenumi}{\roman{enumi}.} %  (i., ii., iii.,...)
%\renewcommand{\labelenumi}{\Roman{enumi}.} %  (I., II., III.,...)
%\renewcommand{\labelenumi}{\alph{enumi}.}   % (a., b., c.,...)
\renewcommand{\labelenumi}{(\alph{enumi})} % [(a), (b), (c),...]
%\renewcommand{\labelenumi}{\Alph{enumi}.}  %  (A., B., C.,...)
\lstdefinestyle{mystyle}{
    backgroundcolor=\color{backcolour},   
    commentstyle=\color{codegreen},
    keywordstyle=\color{magenta},
    numberstyle=\tiny\color{codegray},
    stringstyle=\color{codepurple},
    basicstyle=\ttfamily\small,
    breakatwhitespace=false,         
    breaklines=true,                 
    captionpos=b,                    
    keepspaces=true,                 
    numbers=left,                    
    numbersep=5pt,                  
    showspaces=false,                
    showstringspaces=false,
    showtabs=false,                  
    tabsize=4,
    frame=single % Opcional: añade un marco alrededor
}

\newtheorem{theorem}{Theorem}[chapter]
\newtheorem{teorema}[theorem]{Teorema}
\newtheorem{lem}{Theorem}[chapter]
\newtheorem{lema}[lem]{Lema}
\newtheorem{prop}{Theorem}[chapter]
\newtheorem{proposition}[prop]{Proposición}
\newtheorem{coro}{Theorem}[chapter]
\newtheorem{corolario}[coro]{Corolario}
\newtheorem{exam}{Theorem}[chapter]
\newtheorem{example}[exam]{Ejemplo}
\newtheorem{test}{Theorem}[chapter]
\newtheorem{prueba}[test]{Prueba}
\newtheorem{defi}{Theorem}[chapter]
\newtheorem{definition}[defi]{Definición}
\newtheorem{obs}{Theorem}[chapter]
\newtheorem{observacion}[obs]{Observación}

\newcommand{\keywords}[1]{\par\addvspace\baselineskip
\noindent\keywordname\enspace\ignorespaces#1}
\renewcommand{\abstractname}{\prefacesection{Resumen}}
\renewcommand{\listtablename}{Índice de Tablas}
\renewcommand{\tablename}{Tabla}
\renewcommand{\refname}{Bibliografía}
\newcommand{\boxtheorem}{\hfill $\Box$}
%%%%%%%%IEEE Palabras Claves
%\renewcommand{\IEEEkeywords}{\textbf{\emph{Palabras Clave---}}}


\newcommand{\qed}{\nobreak \ifvmode \relax \else
      \ifdim\lastskip<1.5em \hskip-\lastskip
      \hskip1.5em plus0em minus0.5em \fi \nobreak
      \vrule height0.75em width0.5em depth0.25em\fi}

\author{César Cerda Rifo}
\title{Detección y caracterización de exoplanetas mediante un híbrido de los algoritmos rsm-paco en observación directa\\
    Curso de AnteProyecto de Título}
\degree{Ingeniería Civil Informática}
%\degree{Ingeniería en Ejecución en Computación e Informática}


\supervisor{Tatiana Gutiérrez-Bunster y Alejandro Valdés Jiménez}
\institution{Universidad del Bío-Bío, Chile}
\faculty{Facultad de Ciencias Empresariales}
\department{Departamento de Sistemas de Información}


\begin{document}
\hyphenation{com-pu-ta-dor}
\cleardoublepage
\pagenumbering{roman}
\setcounter{page}{1}
%% En esta secci'on se describe la estructura del documento de la tesis.
%% Consulta los reglamentos de tu universidad para determinar el orden
%% y la cantidad de secciones que debes de incluir

%% # Portada de la tesis #
%% Mirar el archivo "titlepage.tex" para los detalles.

\include{titlepage}

%% # Prefacios #
%% Por cada prefacio (p.e. agradecimientos, resumen, etc.) crear
%% un nuevo archivo e incluirlo aqu'i


%\include{resumen}
%\input{capitulos/resumen}
%\input{capitulos/dedicatoria}
%\input{capitulos/acronimos.tex}

\tableofcontents
%\listoffigures
\renewcommand{\listfigurename}{Índice de Figuras}
\listoffigures
\renewcommand{\listtablename}{Índice de Tablas}
\listoftables

\cleardoublepage
\pagenumbering{arabic}
\setcounter{page}{1}

\acrodef{VB}{Visula Basic}
\acrodef{FP}{Formación Profesional}
\acrodef{SIDA}{Síndrome de Inmunodeficiencia Adquirida}

\chapter{Presentación del área}  \label{cap:introduccion}
En el campo de la astronomía moderna, el procesamiento de imágenes de alto contraste representa un desafío computacional único que combina técnicas avanzadas de computación de alto rendimiento con análisis estadístico especializado \cite{Follette_2023}. Este tipo de procesamiento se aplica cuando necesitamos detectar objetos extremadamente tenues como exoplanetas -planetas que orbitan estrellas fuera del sistema solar- junto a fuentes de luz intensamente brillantes como lo son sus estrellas anfitrionas.\\ Para ello existen diversos métodos o técnicas para detectar y caracterizar exoplanetas, entre estos se encuentra la imagen directa.
Esta técnica busca capturar directamente la luz reflejada por los planetas de las estrellas que orbitan, lo que permitiría caracterizar sus atmósferas y condiciones físicas. Sin embargo, el problema fundamental radica en que estos objetos son extraordinariamente tenues en comparación con sus estrellas anfitrionas, típicamente entre 1 y 100 millones de veces más débiles  \cite{NAP25187}.\\
Este desbalance de intensidades plantea retos computacionales que han motivado el desarrollo de algoritmos especializados durante la última década. La dificultad principal no reside solamente en la debilidad de la señal, sino en la naturaleza estructurada del ruido dominante en estas imágenes. Los telescopios modernos terrestres equipados con óptica adaptativa, como el Very Large Telescope (VLT) \citep{ESO_AdaptiveOptics_ELT}, pueden corregir en tiempo real las distorsiones causadas por la atmósfera terrestre, pero aún dejan patrones residuales complejos conocidos como "speckles". Estos artefactos presentan características espaciales y temporales que los hacen particularmente engañosos, ya que pueden persistir durante varias observaciones y mimetizar perfectamente la apariencia esperada de un exoplaneta. Aquí es donde la computación juega un papel crucial ya que los algoritmos especializados deben limpiar las imágenes de ruido sistemático, identificar posibles planetas entre los artefactos restantes y hacer esto de manera eficiente para procesar miles de imágenes.\\
\newpage
\begin{figure}[h]
    \centering
    \includegraphics[height=6cm]{challenge_fig2.001}
    \caption{Representación esquemática del flujo de procesamiento de datos de imagen de alto contraste. Caso de un cubo de datos de la estrella HR8799 (Exoplanet Imaging Data Challenge 1, Copyright (c) 2018 Carlos Alberto Gomez Gonzalez, MIT License. \cite{ExoplanetImagingChallenge}).}
    \label{edicraw}
\end{figure}


El proceso, como se ve en la figura \ref{edicraw}, comienza con imágenes crudas en formato FITS \citep{refId0} -el formato estándar en astronomía, como JPEG pero con datos del telescopio añadidos- las cuales son esencialmente matrices numéricas multidimensionales. Estas imágenes pasan por una cadena de preprocesamiento básico donde se hace una limpieza de las imágenes, por ejemplo, corrigiendo los píxeles muertos o defectuosos. Luego se realiza el alineamiento con precisión subpixel usando correlación cruzada y  técnicas especializadas de imagen diferencial  como lo son las secuencias ADI (Angular Differential Imaging) \citep{refId1} que 
consisten en múltiples imágenes tomadas mientras el telescopio gira, lo que hace que los exoplanetas (objetos móviles) cambien de posición, mientras que el ruido instrumental (speckles) permanece fijo, lo que permite separar señales planetarias de artefactos mediante alineamiento y resta; y secuencias SDI (Spectral Differential Imaging) que capturan la misma escena en diferentes longitudes de onda o colores, aprovechando que los planetas tienen firmas espectrales distintas al ruido. Al comparar estas imágenes, se suprimen los artefactos que varían con el color, mejorando la detección. Estas técnicas generan cubos de datos 3D (tiempo × espacio × longitud de onda) donde cada frame pesa entre 16MB y 64MB aproximadamente y estos cubos de datos son sometidos a técnicas de reducción de ruido y mejora de contraste \citep{Samland_2022}. Esto nos plantea desafíos computacionales específicos como lo son el manejo de datos masivos, ya que una sola observación puede generar más de 100GB de datos y requieren procesamiento para imágenes con una resolución mayor o igual a 2K×2K \citep{refId2}. Además del manejo de cálculos matemáticos complejos ya que se realizan operaciones matriciales masivas que deben ejecutarse eficientemente en GPU.

Actualmente, los observatorios como VLT y el James Webb Space Telescope (JWST) generan volúmenes de datos que exceden los 250GB por noche \cite{article}, demandando algoritmos que no solo sean precisos, sino también computacionalmente eficientes y escalables.



\section{Presentación de los procesos del ámbito del proyecto}\label{procesos}

La investigación se centra en dos procesos computacionales fundamentales para la detección y caracterización de exoplanetas. El primero es la reducción de speckles mediante PACO (Principal Component Analysis plus Covariance) \citep{paco}. Los componentes principales son vectores ortogonales que representan los patrones dominantes de variabilidad en los datos, obtenidos mediante descomposición matricial de las imágenes de entrada. PACO opera en tres etapas principales:
\begin{enumerate}
    \item Descomposición espectral de la secuencia de imágenes usando PCA (Análisis de componentes principales) para identificar los patrones dominantes de ruido, PCA toma muchas imágenes con ruido e identifica patrones repetitivos para luego separar esos patrones molestos(componentes principales) de lo que cambia entre imágenes.

    \item Modelado de covarianza local que analiza las variaciones espaciales del ruido residual.

    \item Reconstrucción y sustracción selectiva de los artefactos. Este proceso, que tradicionalmente se implementa en CPU, requiere operaciones matriciales masivas sobre conjuntos de datos que pueden superar los 100GB, presentando desafíos significativos de paralelización y gestión de memoria. 
\end{enumerate}
El segundo proceso clave es la detección bayesiana mediante RSM (Regime Switching Model) \cite{rsm}, que aborda el problema desde una perspectiva estadística rigurosa. RSM modela explícitamente la probabilidad de que un píxel dado contenga solo ruido residual o una señal planetaria genuina, ofreciendo ventajas teóricas como cuantificación natural de incertidumbre y mayor robustez frente a falsos positivos. Sin embargo, su implementación clásica mediante Markov Chain Monte Carlo (MCMC) resulta computacionalmente muy compleja, requiriendo días de procesamiento por imagen en configuraciones estándar.

	% Primera tabla
	\begin{table}[H] 
		\centering
		\small
		\begin{tabular}{p{5cm}p{5cm}p{5cm}}
			\toprule
			\textbf{Concepto Astronómico} &  
                \textbf{Técnica Computacional} &  
                \textbf{Complejidad}\\
			\midrule
			\textbf{Supresión de speckles}  & PCA + Covarianza     & \(O(n^2 log n)\)     \\
			\midrule
			\textbf{Detección de señales} & Modelado        Bayesiano & \(O(n^3)\)     \\
			\midrule
			\textbf{Mejora de contraste} & Filtrado espacial & \(O(n^2)\)     \\
			\bottomrule
		\end{tabular}
        \caption{Correspondencia entre Conceptos y Técnicas Computacionales}
        \label{tablaejemplo}
	\end{table}
	
La interacción óptima entre estos dos procesos - limpieza inicial con PACO seguida de análisis estadístico con RSM - constituye el núcleo de la investigación, buscando superar las limitaciones actuales en sensibilidad y escalabilidad mediante innovaciones algorítmicas y paralelización CPU/GPU. La Tabla \ref{tablaejemplo} relaciona algunos conceptos y operaciones astronómicas junto con la técnica computacional que utilizan y las complejidades temporales de cada una.\\



\section{Descripción del Problema}\label{problema}
La detección de exoplanetas mediante imágenes directas es un desafío computacional relevante, ya que implica identificar señales extremadamente débiles en imágenes dominadas por ruido estructurado. Para lograr esto, se utilizan tecnologías como óptica adaptativa (para corregir en tiempo real las distorsiones causadas por la atmósfera) y coronógrafos (que bloquean la luz directa de la estrella para mejorar el contraste).

El principal obstáculo desde el punto de vista del procesamiento de imágenes radica  en separar las señales planetarias que son entre $10^{-6}$ a $10^{-4}$ veces más debiles que su estrella asociada, del ruido estructurado, tambien llamado speckless o speckle noise. Éstos son artefactos ópticos que aparecen como patrones de interferencia en las imagenes  y son generados por imperfecciones instrumentales -por ejemplo, los espejos telescópicos- y turbulencia atmosferica residual. Además los speckless no son estacionarios, es decir, varían en tiempo/espacio; y su intensidad puede llegar a ser hasta \(\approx10^6 \) veces más fuerte que la señal objetivo.\\
Para abordar este problema, se han desarrollado algoritmos como PACO y RSM, que aplican técnicas de procesamiento estadístico y aprendizaje automático para modelar y separar la señal del ruido. No obstante, ambos presentan limitaciones:

PACO utiliza PCA para modelar el ruido local en las imágenes y detectar anomalías. Si bien es eficiente y se puede paralelizar en CPU, sufre pérdida de sensibilidad en zonas cercanas a la estrella, menores a 0.3 arcosegundos, lo que equivale a aproximadamente 3 unidades astronómicas (3 veces la distancia entre la tierra y el sol) o 10 parsec (3.2616 años luz),  debido a un exceso de suavizado que elimina tanto el ruido como las señales débiles \cite{complexpaco}. Además, requiere un ajuste manual del número de componentes de PCA, lo que complica la optimización de hiperparámetros.\\
RSM adopta un enfoque bayesiano, modelando cada píxel como una mezcla probabilística entre señal y ruido, y aplicando inferencia con MCMC. Este enfoque es más preciso para distinguir verdaderas señales planetarias, pero su alto costo computacional(<72h/objetivo en CPU) para surveys masivos \citep{complexrsm} como por ejemplo el ELT(Extremely Large Telescope) \(\approx10^4 \) estrellas objetivo- ya que el MCMC requiere millones de iteraciones/píxel (\((O(n^3)\) en CPU), lo que lo hace no escalable e inviable para imágenes de 2Kx2K que son tipicas en astronomía.\\
Por lo tanto, se identificó una brecha técnica importante, haciendo falta un framework que preprocese eficientemente (acelerando PACO en GPU) y clasifique señales con modelos bayesianos escalables como RSM que utiliza el método de inferencia tradicional pero reemplazandolo por Inferencia Variacional


\section{Propuesta de Solución}
En base a las brechas técnicas identificadas, se clarifica la necesidad de métodos que sean computacionalmente eficientes y estadísticamente robustos. En particular, se propone una solución híbrida que combina dos enfoques complementarios: PACO, que permite suprimir ruido estructurado de forma eficiente, y RSM, que permite detectar patrones estadísticamente significativos en presencia de señales débiles. Mientras que PACO destaca por su velocidad y escalabilidad, RSM ofrece una mayor sensibilidad en la detección de señales planetarias, aunque a un mayor costo computacional. Se busca integrar ambas metodologías en un pipeline modular que aprovecha la capacidad de cómputo paralelo de las GPU modernas para procesar grandes volúmenes de datos sin sacrificar precisión.\\

\begin{figure}[h]
    \centering
    \includegraphics[height=8cm]{capitulos/diagramapipeline.png}

    \caption{Diagrama del pipeline propuesto.}
    \label{diagpipe}
\end{figure}


La arquitectura propuesta  en la figura \ref{diagpipe} consta de tres etapas principales. En primer lugar, se realiza un preprocesamiento standard donde se realizará la calibración y normalización de datos, alineamiento subpixel, reducción de ruido instrumental, etc. y luego se pasará a la segunda etapa donde se encuentra el procesamiento acelerado por GPU, donde se implementará PACO utilizando CUDA para optimizar operaciones locales de covarianza sobre parches de imagen. Esto permite limpiar las imágenes sin eliminar las posibles señales planetarias. Al mismo tiempo, en paralelo en esta segunda etapa, se introducirá un modelo generativo con inferencia variacional que reemplazará técnicas tradicionales como MCMC, permitiendo entrenar RSM de forma escalable mediante optimización basada en gradientes. Este modelo incorpora información instrumental y restricciones físicas para mejorar su capacidad de generalización.\\ Finalmente, la etapa de postprocesamiento se encargará de la clasificación probabilística de detecciones, generación de mapas de incertidumbre y filtrado espacial adaptativo. El sistema completo operará sobre flujos de datos paralelos -imágenes, parámetros instrumentales, además de modelos preentrenados- y estará optimizado para trabajar con precisión numérica mixta (FP16 y FP32), lo cual mejorará el rendimiento sin afectar la calidad del análisis.\\\\



Esta necesidad aumentará con la próxima generación de telescopios como el ELT, que generará cientos de gigabytes de datos por noche. Esta propuesta busca desarrollar una solución híbrida que combine lo mejor de ambos enfoques: la velocidad de PACO y la precisión de RSM, optimizada para hardware moderno como GPUs NVIDIA.\\

Para evaluar el método propuesto, se utilizarán datos sintéticos generados con el paquete pynpoint \cite{pynpoint}, que simula condiciones reales de observación. Esto nos entregará datos de referencia conocidos y verificados que nos permitan evaluar objetivamente el rendimiento de un algoritmo. Además, también utilizarán datos reales, imágenes públicas de SPHERE/VLT \cite{ESO_SPHERE_InstrumentPage} y JWST \cite{WebbTelescope_Images}, donde existen candidatos conocidos para validación.\\Las métricas claves serán:
\begin{itemize}
    \item Tasa de recuperación (recall) para señales débiles
    \item Precisión en la posición de planetas detectados
    \item Tiempo de procesamiento por imagen
    \item Uso de memoria en GPU
\end{itemize} 
El impacto esperado sería permitir la detección de planetas terrestres alrededor de estrellas cercanas, objetivo clave de misiones como HARMONI/ELT (High Angular Resolution Monolithic Optical and Near-infrared Integral field spectrograph), establecer nuevos benchmarks para inferencia bayesiana en imágenes y desarrollar técnicas aplicables a otros campos, por ejemplo, medicina, que trabajen con ruido estructurado.
\subsection{Objetivo General}
Desarrollar e implementar un pipeline computacional híbrido que combine los algoritmos PACO y RSM para la detección de exoplanetas en imágenes de alto contraste, acelerado para arquitecturas GPU y que supere las limitaciones de sensibilidad y escalabilidad de los métodos actuales.

\subsection{Objetivos Específicos}
El presente proyecto de investigación se estructura en torno a cuatro objetivos específicos claramente definidos y medibles:
\begin{itemize}
    \item
    \textbf{{Objetivo Específico 1: Revisión bibliográfica e implementación base:}}\\Revisar exhaustivamente la bibligrafía existente para luego implementar cada algoritmo o método individualmente y generar un benchmark a utilizar para comparar con la implementación del pipeline.
    \item \textbf{{Objetivo Específico 2: Acelerar el algoritmo PACO mediante computación GPU:}}
    Implementar una versión acelerada del algoritmo PACO en CUDA que reduzca el tiempo de procesamiento de imágenes astronómicas cpm comparado con la implementación CPU de referencia, manteniendo la precisión numérica y validando resultados con datos sintéticos.
    \item \textbf{Objetivo Específico 3: Desarrollar RSM con inferencia variacional escalable:}
Reducir la complejidad computacional del algoritmo RSM de O(n³) a O(n) mediante la implementación de inferencia variacional en PyTorch, logrando procesar imágenes de 2K×2K en menor tiempo mientras se mantiene una precisión equivalente al 95\% del método MCMC tradicional( debido a que MCMC a diferencia de VI es asintóticamente exacto), haciendolo escalable.
    \item \textbf{Objetivo Específico 4: Integrar ambos algoritmos en un pipeline híbrido optimizado:}
Desarrollar un sistema integrado RSM-PACO que mejore las métricas de detección (FDR/MDR)  comparado con el mejor método individual, implementando gestión eficiente de memoria GPU y flujo de datos optimizado para procesamiento de surveys masivos.
    \item \textbf{Objetivo Específico 5: Validar el método propuesto con datos astronómicos reales:}
Demostrar las ventajas del pipeline híbrido mediante validación con datos sintéticos controlados y observaciones reales de SPHERE/VLT, comparando cuantitativamente contra métodos baseline (PCA, LOCI, PACO standalone) usando métricas estándar en astronomía.
\end{itemize}

\subsection{Descripción de actividades}
Las actividades a realizar se organizan en etapas asociadas a cada objetivo, cada una con tareas concretas y entregables medibles, alineadas con el cronograma detallado del proyecto:

\subsubsection{Etapa 1: Preparación}
Las actividades de preparación, asociadas al objetivo 1 incluyen:
\begin{itemize}
    \item Revisión bibliográfica exhaustiva de RSM y PACO
    \item Aprendizaje intensivo de CUDA y programación en GPU
    \item Setup del entorno de desarrollo (CUDA, PyTorch, herramientas astronómicas)
    \item Familiarización con datos FITS y procesamiento astronómico
    \item Estudio de métricas de evaluación específicas del dominio
    \item Implementación de carga y preprocesamiento de datos FITS
    \item Implementación de algoritmos de alineamiento subpíxel
    \item Prototipo de PCA básico en CPU como referencia
\end{itemize}

\subsubsection{Etapa 2: Desarrollo nuevos algoritmos}
Esta etapa se subdivide en dos subetapas correspondientes a los objetivos 2 y 3:
\begin{itemize}
\item Subetapa 1, desarrollo y mejoras PACO
    \begin{itemize}
        \item Implementación de PACO base repositorio original
        \item Implementación de PACO en libreria/paquete VIP
        \item Implementación acelerada en CPU y pruebas
        \item Implementación de kernels CUDA para covarianza local
        \item Implementación de PACO acelerado en GPU
        \item Medición métricas de desempeño respecto a algoritmo base
    \end{itemize} 
    \item Subetapa 2, desarrollo y mejoras RSM
    \begin{itemize}
        \item Implementación base RSM con MCMM
        \item Medición rendimiento version base, identificación de problemas de implementación y mejoras al algoritmo.
        \item Desarrollo de RSM básico con inferencia variacional
        \item Medición métricas de desempeño respecto a algoritmo base
    \end{itemize}
    
\end{itemize}

\subsubsection{Etapa 3: Implementación del Pipeline Híbrido}
Las actividades de integración avanzada asociadas al objetivo 4 incluyen:
\begin{itemize}
    \item Integración completa de PACO optimizado para GPU
    \item Optimización avanzada de manejo de memoria GPU
    \item Desarrollo del pipeline híbrido RSM-PACO
    \item Calibración y ajuste con datos sintéticos
    \item Implementación de pruebas unitarias y debugging
\end{itemize}

\subsubsection{Etapa 4: Validación y Evaluación}
Las actividades de validación científica asociadas al objetivo 5 comprenden:
\begin{itemize}
    \item Implementación de métricas FDR/MDR y análisis ROC
    \item Benchmarks de rendimiento GPU vs CPU
    \item Validación con datos reales de telescopios
    \item Análisis de sensibilidad y robustez del método
\end{itemize}

\subsubsection{Etapa 5: Análisis Comparativo y Documentación}
Las actividades de finalización incluyen:
\begin{itemize}
    \item Comparación sistemática con técnicas estado del arte
    \item Documentación técnica completa del código
    \item Redacción del informe final de tesis
    \item Preparación de presentación y defensa
\end{itemize}
\newpage

\section{Descripción de los Aspectos Fundamentales de la Metodología a utilizar en la Investigación}
La Tabla \ref{tabla aspectos} caracteriza el proyecto a desarrollar, en distintos aspectos:
{
\sloppy
\begin{table}[H]
\begin{tabular}{|m{3cm}|m{1.5cm}|m{10.5cm}|}
    \hline
    Ítem & Nivel & Descripción \\
    \hline
    Tipo de investigación &  Aplicada  & Investigación orientada a resolver un problema, técnico-científico mediante el desarrollo de un nuevo enfoque computacional validado sobre datos reales y simulados.\\
    \hline
    Conocimiento del tema & Medio & Se dispone de comprensión sólida sobre el 
    proceso de formación de imágenes astronómicas 
    y los desafíos del ruido estructurado, pero
    no hay un método establecido que
    resuelva simultáneamente precisión y eficiencia.\\

    \hline
    Innovación metodológica  &  Alta & Se propone una combinación original de dos
    enfoques (PACO y RSM), usando inferencia 
    variacional y computación paralela,
    lo que no ha sido explorado
    en esta forma dentro del campo.\\ 

    \hline
    Reproducibilidad &  Alta & Todos los módulos del sistema serán
    implementados utilizando herramientas
    abiertas (CUDA, PyTorch, Pyro)
    y sobre datasets públicos o generados con
    software  reproducible (PynPoint).\\ 

    \hline
    Complejidad computacional & Alta & El método incluye operaciones de álgebra lineal 
    mixta intensivas (PCA, covarianzas locales),
    entrenamiento bayesiano, manejo de
    precisión numérica y sincronización
    múltiples flujos de datos.\\
    \hline
    Modularidad del sistema & Alta & El sistema se divide en tres módulos 
    (preprocesamiento, inferencia, 
    postprocesamiento) que pueden desarrollarse, 
    evaluarse y optimizarse de manera independiente. \\
    \hline
    Escalabilidad &  Alta & El enfoque está diseñado para 
    escalar en volumen de datos 
    y adaptarse a futuros instrumentos 
    de mayor resolución, como el ELT.\\
    \hline
\end{tabular}
\caption{Tabla comparativa de aspectos a considerar/desarrollar en la investigación}
   \label{tabla aspectos}
\end{table}

Dada la naturaleza exploratoria, computacional e incremental de la investigación, se utilizará una metodología iterativa basada en el desarrollo por módulos. Cada componente (PACO-GPU, RSM-VI, pipeline integrado) será diseñado, implementado, validado y optimizado de manera independiente. Este enfoque permite evaluar el impacto de cada parte del sistema sobre los objetivos generales, y facilita la detección temprana de limitaciones o cuellos de botella tanto computacionales como estadísticos.
}
%\section{Composición del informe}
%El presente trabajo se encuentra dividido en xx capítulos. A continuación se describe brevemente el contenido de cada uno de ellos.

%\begin{enumerate}
    %\item \textbf{Introducción:} texto.

    %\item \textbf{Bases Teóricas:}texto.
    
    %\item \textbf{Estado del Arte:} texto.
    
    %\item \textbf{Desarrollo del trabajo}texto.
    
   % \item \textbf{Resultados} texto.
    
    %\item \textbf{Conclusiones:} texto.
%\end{enumerate}
%Además, al final del informe se adjuntan las referencias con los artículos utilizados en el proceso de investigación.


\chapter{Bases Teóricas y Antecedentes en los que se basa la Investigación} \label{cap:marco_teorico}
En este capítulo se presentan las bases teóricas fundamentales bajo las cuales se sustenta la investigación.

\section{Fundamentos de Imagen Directa en Astronomía}
La imagen directa de exoplanetas plantea un problema único en visión por computador: detectar fuentes puntuales con relaciones señal-ruido (SNR) extremadamente bajas (típicamente <5, significando que la señal del exoplaneta es menos de cinco veces mayor que el nivel promedio de las fluctuaciones aleatorias del fondo) en presencia de patrones de ruido altamente estructurados. Desde la perspectiva informática, esto implica:



\subsection{Formato FITS y Preprocesamiento Inicial}El formato FITS (Flexible Image Transport System) constituye el estándar fundamental en astronomía para el almacenamiento y transmisión de datos científicos. A diferencia de formatos convencionales como JPEG o PNG, FITS preserva metadatos observacionales esenciales (coordenadas celestes, tiempo de exposición, parámetros instrumentales) junto con valores numéricos de alta precisión (float32/float64) sin aplicar compresión con pérdidas. Cada archivo FITS contiene tres componentes (ver Ecuación \ref{eqfits}) principales: un header con metadatos en texto ASCII, una o más extensiones con datos multidimensionales (típicamente cubos 3D [tiempo, x, y]), y opcionalmente tablas binarias. En el pipeline, esta estructura permite manejar eficientemente secuencias de cientos de imágenes de 4K×4K píxeles, donde cada píxel representa el flujo fotónico capturado por el detector. La etapa de preprocesamiento incluye corrección de píxeles defectuosos mediante máscaras predefinidas, sustracción del dark current (ruido térmico del sensor), y normalización por flat field (variaciones en la sensibilidad del detector), operaciones que se implementan mediante álgebra matricial vectorizada en GPU para procesar terabytes de datos.
\begin{equation}\label{eqfits}
    X \in \mathbb{R}^{T \times H \times W}
\end{equation}donde:
\begin{itemize}
    \item T: Número de frames (típicamente 100-1000)
    \item H×W: Resolución espacial (2048×2048 en SPHERE)
    \item Cada píxel almacena flujo fotónico en float32
\end{itemize}


A continuación se muestra el Algoritmo \ref{alg:fits}, ejemplo de cómo cargar un cubo de datos astronómicos en formato FITS utilizando la biblioteca \texttt{astropy.io}:\\

\begin{algorithm}[H]
\caption{Lectura de archivo FITS en Python}
\label{alg:fits}
\KwData{Archivo FITS con observaciones astronómicas}
\KwResult{Cubo de datos como arreglo NumPy}
\SetKwInOut{Input}{input}
\SetKwInOut{Output}{output}

\Input{observacion.fits}
\Output{cube con forma (T, H, W)}

\Begin{
    \texttt{from astropy.io import fits}\;
    \texttt{hdul = fits.open('observacion.fits')}\;
    \texttt{cube = hdul[0].data}\;
    \tcc{cube tiene forma (T, H, W) -> tiempo, alto, ancho}
}
\end{algorithm}

Este código abre un archivo FITS y extrae un cubo de imágenes, que típicamente representa una secuencia temporal de observaciones en 2D. La variable \texttt{cube} es un arreglo tridimensional con forma \((T, H, W)\), donde \(T\) es el número de imágenes (o tiempo), y \(H\) y \(W\) son las dimensiones espaciales (alto y ancho).



\subsection{Alineamiento Subpíxel y Registro de Imágenes}
El alineamiento subpíxel es un proceso computacional crítico que corrige desplazamientos mínimos entre frames causados por vibraciones mecánicas, turbulencia atmosférica residual, y deriva del telescopio. Esta técnica opera calculando la correlación cruzada normalizada entre imágenes consecutivas en el dominio de Fourier, donde los desplazamientos se manifiestan como fases lineales en la transformada, ver Ecuación \ref{eqsubp}. Mediante interpolación parabólica del pico de correlación, alcanzamos precisión de 0.1-0.01 píxeles, esencial para evitar que los speckles (artefactos ópticos que imitan señales planetarias) se difuminen durante el apilado. Puede ser implementado usando \textit{hase\_cross\_correlation}
 de scikit-image (ver Algoritmo \ref{alg:alignment}) optimizado con Numba. La precisión subpíxel es particularmente crucial cuando se trabaja con separaciones angulares menores a 0.5 arcosegundos, donde un error de alineación de 0.1 píxeles puede significar perder hasta el 40\% del flujo planetario.

\begin{equation}\label{eqsubp}
 (\Delta x, \Delta y) = \underset{\delta_x, \delta_y}{\mathrm{argmax}} \left| \mathcal{F}^{-1} \left\{ \mathcal{F}\{I_1\} \cdot \mathcal{F}^*\{I_2\} \right\} \right|
\end{equation} 

\begin{algorithm}[H]
\caption{Alineamiento Subpíxel}
\label{alg:alignment}
\KwData{Cubo de imágenes astronómicas}
\KwResult{Lista de desplazamientos entre frames}

\Begin{
    \texttt{from skimage.registration import phase\_cross\_correlation}\;
    \texttt{shifts = [phase\_cross\_correlation(ref, img) for img in cube]}\;
}
\end{algorithm}
El Algoritmo implementa la función crítica de alineamiento subpíxel mediante correlación cruzada en el dominio de la frecuencia, utilizando la biblioteca \texttt{scikit-image} optimizada con técnicas de interpolación parabólica. La función \texttt{phase\_cross\_correlation} calcula el desplazamiento óptimo entre una imagen de referencia y cada frame del cubo de datos, aprovechando las propiedades de la transformada de Fourier para detectar traslaciones con precisión subpíxel del orden de 0.01-0.1 píxeles.

La importancia del alineamiento subpíxel radica en que, sin esta corrección, los patrones de speckle se difuminan durante el procesamiento estadístico, resultando en una pérdida significativa de la señal planetaria. En particular, para separaciones angulares menores a 0.5 arcosegundos, un error de alineación de apenas 0.1 píxeles puede ocasionar una pérdida del 40\% del flujo fotónico del exoplaneta objetivo, comprometiendo gravemente la sensibilidad del método de detección.


\section{Algoritmos Clave}
Sección que presenta teóricamente los algoritmos a utilizar en el proyecto y que darán la base del pipeline a implementar.
\subsection{Principal Component Analysis plus Covariance -- PACO}El núcleo de nuestro pipeline de reducción de ruido es PACO (Principal Component Analysis plus Covariance), un algoritmo híbrido que combina descomposición matricial con modelado estadístico espacial. En su primera fase, se aplicará PCA (Análisis de Componentes Principales) mediante descomposición en valores singulares (SVD) al cubo de imágenes alineadas, ver Ecuación \ref{eqpca1}, identificando los k patrones dominantes de speckles. Matemáticamente, esto corresponde a factorizar la matriz de datos X (de tamaño n\_frames × n\_píxeles) como X = U\(\Sigma\)\(V^T\), donde las columnas de V contienen los componentes principales (patrones de speckle) y \(\Sigma\) su importancia relativa. La selección del número óptimo de componentes (k) sigue el criterio de Marchenko-Pastur -describe la distribución asintótica de los valores propios (o valores singulares) de matrices aleatorias grandes- evitando tanto sobresustracción (que borraría planetas) como subsustracción (dejando ruido residual). La segunda fase modelaría variaciones locales mediante matrices de covarianza, ver Ecuación \ref{eqpca2}, calculadas en ventanas deslizantes de 9×9 píxeles, capturando así la no estacionariedad del ruido. Se planea implementar este doble proceso en CUDA (Compute Unified Device Architecture) y usando cuBLAS (CUDA Basic Linear Algebra Subprograms), librería de bajo nivel escrita en C/C++ desarrollada por NVIDIA, para la SVD y kernels personalizados para la covarianza, logrando aceleración respecto a versiones CPU. Su implementación se puede ver en el Algoritmo \ref{alg:cuda_cov}. \\\\
PCA: Identifica patrones globales de ruido mediante SVD
\begin{equation}\label{eqpca1}
X = U \Sigma V^T \quad \Rightarrow \quad X_{\text{recon}} = \sum_{i=1}^k \sigma_i \mathbf{u}_i \mathbf{v}_i^T
\end{equation}
Covarianza local: Modela variaciones espaciales en ventanas 9×9 píxeles

\begin{equation}\label{eqpca2}
\mathbf{C}_{pq} = \frac{1}{N-1} \sum_{i \in \mathcal{N}} (x_i - \bar{x})(x_i - \bar{x})^T
\end{equation}

\begin{algorithm}[H]
\caption{Kernel CUDA para Covarianza Local}
\label{alg:cuda_cov}
\KwData{Matrices de entrada y salida en GPU}
\KwResult{Cálculo de covarianza local optimizado}

\Begin{
    \tcc{Configuración del kernel CUDA}
    \texttt{\_\_global\_\_ void cov\_kernel(float* input, float* output) \{}\;
    \Indp
        \texttt{\_\_shared\_\_ float tile[32][32]}\;
        \texttt{int idx = blockIdx.x * blockDim.x + threadIdx.x}\;
        \texttt{int idy = blockIdx.y * blockDim.y + threadIdx.y}\;
        \BlankLine
        \tcc{Cargar datos en memoria compartida}
        \texttt{if (idx < width \&\& idy < height) \{}\;
        \Indp
            \texttt{tile[threadIdx.y][threadIdx.x] = input[idy * width + idx]}\;
        \Indm
        \texttt{\}}\;
        \texttt{\_\_syncthreads()}\;
        \BlankLine
        \tcc{Calcular covarianza local en ventana 9x9}
        \texttt{float covar = 0.0f}\;
        \texttt{for (int i = -4; i <= 4; i++) \{}\;
        \Indp
            \texttt{for (int j = -4; j <= 4; j++) \{}\;
            \Indp
                \texttt{covar += computeCovariance(tile, i, j)}\;
            \Indm
            \texttt{\}}\;
        \Indm
        \texttt{\}}\;
        \BlankLine
        \tcc{Escribir resultado optimizado}
        \texttt{if (idx < width \&\& idy < height) \{}\;
        \Indp
            \texttt{output[idy * width + idx] = covar}\;
        \Indm
        \texttt{\}}\;
    \Indm
    \texttt{\}}\;
}
\end{algorithm}

El Algoritmo \ref{alg:cuda_cov} implementa el cálculo de covarianza local optimizado en GPU basado en \cite{inproceedings}, \cite{hangün2019performancecomparisonopencvbuilt}, \cite{cuda_large_scale_gpu_mlplus2025} y  \cite{nvidia_cuda_best_practices_2025} siendo ideal para operaciones de matrices de covarianza y procesamiento de imágenes. Dado que la técnica de sliding window es recomendada para matrices de covarianza \cite{iwakura2008sliding}. El kernel utiliza memoria compartida  \cite{young2008image_cuda}(\texttt{\_\_shared\_\_}) para poder dividir la matriz en submatrices más pequeñas y hacer el proceso más eficiente minimizando accesos a la memoria global \cite{dhanush2024masteringcuda}, organizando los datos en bloques de 32×32 \cite{nvidia_cuda_best_practices_2025} threads ya que los SMs (streaming multiprocessors) tienen recursos limitados, 16384 registros de 32 bits que guardan datos temporales de los hilos \citep{adaptableCuda_slidingwindow_neu_rcl} y para maximizar el rendimiento varios bloques de hilos pueden ejecutarse simultáneamente en un mismo SM. Al ejecutarse un bloque de hilos, CUDA los divide en grupos de 32 hilos llamados warps, por lo tanto, para que los warps estén siempre completamente llenos y se maximice la eficiencia, además, escoger bloques de 32x32 es óptimo ya que permite disminuir la complejidad computacional usando parches de subventanas aplicables a PCA y como el máximo número de hilos por bloque es 1024 (32x32) según \cite{nvidia_cuda_best_practices_2025}, se escogió ese valor de bloques para que calce directamente con los parches de covarianza y subparches a nivel teórico en base a \citep{Kwatra2010Fast}, que procesan ventanas deslizantes de 9×9 píxeles el cual es un valor arbitrario. \\Cada thread calcula la covarianza para su píxel correspondiente, aprovechando la paralelización masiva de la GPU para procesar miles de ventanas simultáneamente.\\Dado que el cálculo de covarianza usando tiles son operaciones que implican escritura y lectura, y que además dependen de los resultados anteriores, siguiendo los principios de Bernstein donde ` ``Dos conjuntos de operaciones pueden ejecutarse en paralelo si y solo si no hay dependencias de flujo, anti-dependencias o dependencias de salida entre ellas'' \cite{4038883}, esto transformaría todo en una operación de tipo stencil donde el stencil sería en este caso el cálculo de covarianza local mediante ventanas y además se cumple la condición de carrera, que ocurre cuando múltiples hilos/procesos acceden simultáneamente a un recurso compartido, en este caso memoria,  sin sincronización, y el resultado final depende del orden impredecible de ejecución. Para poder asegurar la consistencia de los datos y sincronizar los arrays compartidos evitando errores \cite{Markidis2023CUDAStencils}, usamos  \texttt{\_\_syncthreads()} lo que asegura que todos los threads del bloque hayan cargado sus datos antes de proceder con el cálculo \cite{TutorialsPointCUDAThreads}. Cabe recalcar que el algoritmo 3 es una implementación genérica propuesta o una técnica estándar de procesamiento de imágenes en GPU en base a la documentación leída, aplica solo al cálculo de covarianza local que es solamente una parte del algoritmo PACO y no ha sido implementada en el algoritmo completo. En este caso PACO abarca PCA (con cuBLAS) además de  los parches de covarianza y también el algoritmo \ref{alg:paco1}.

Los algoritmos relacionados directamente a PACO descritos en \cite{Flasseur_2018} donde se propusieron son los siguientes:
\newpage
\begin{algorithm}
\caption{PACO detection – Cálculo del mapa S/N para posiciones angulares}
\label{alg:paco1}
\Begin{
    \texttt{for $\phi_0$ in $\mathcal{G}$:} \\
    \quad \texttt{a ← 0} \\
    \quad \texttt{b ← 0} \\
    \quad \texttt{for $\ell = 1$ to $T$:} \\
    \quad\quad \texttt{// Step 1: extraer parches relevantes} \\
    \quad\quad \texttt{$\phi_{\ell} = \mathcal{F}_{\ell}(\phi_0)$} \\
    \quad\quad \texttt{$\mathcal{P}_{\ell} \leftarrow \{r[\phi_{\ell'},\ell']\}_{\ell'=1..T}$} \\
    \quad\quad \texttt{// Step 2: aprender estadísticas del fondo} \\
    \quad\quad \texttt{$\hat{m}_{[\phi_{\ell}]},\,\hat{C}_{[\phi_{\ell}]} \leftarrow$ estimar de $\mathcal{P}_{\ell}$} \\
    \quad\quad \texttt{// Step 3: actualizar acumuladores} \\
    \quad\quad \texttt{$w \leftarrow \hat{C}_{[\phi_{\ell}]}^{-1} \cdot h_{[\phi_{\ell}]}$} \\
    \quad\quad \texttt{$a \leftarrow a + w^{\top} \cdot h_{[\phi_{\ell}]}$} \\
    \quad\quad \texttt{$b \leftarrow b + w^{\top} \cdot (\,r[\phi_{\ell},\ell] - \hat{m}_{[\phi_{\ell}]}\,)$} \\
    \quad \texttt{end for} \\
    \quad \texttt{S/N($\phi_0$) $\leftarrow$ b / $\sqrt{a}$} \\
}
\end{algorithm}
Donde el algoritmo \ref{alg:paco1} calcula la relación señal-ruido (S/N) para cada posición angular $\phi_0$ en el grid $\mathcal{G}$. Para cada $\phi_0$, itera sobre los $T$ frames acumulando dos términos clave: $a$ (normalización de la varianza) y $b$ (señal residual). Primero extrae parches $\mathcal{P}_\ell$ centrados en posiciones transformadas $\phi_\ell = \mathcal{F}_\ell(\phi_0)$, luego estima estadísticas locales del fondo ($\hat{m}_{[\phi_\ell]}$, $\hat{C}_{[\phi_\ell]}$) y finalmente actualiza los acumuladores usando pesos óptimos $w = \hat{C}^{-1}h$. El S/N se obtiene como $b/\sqrt{a}$, maximizando la detección de fuentes débiles en entornos ruidosos.
\newpage


\begin{algorithm}
\caption{Fast PACO detection – Versión acelerada del S/N}
\label{alg:paco2}
\Begin{
    \texttt{for $\phi_0$ in $\mathcal{G}$:} \\
    \quad \texttt{a ← 0} \\
    \quad \texttt{b ← 0} \\
    \quad \texttt{for $\ell = 1$ to $T$:} \\
    \quad\quad \texttt{$\phi_{\ell} = \mathcal{F}_{\ell}(\phi_0)$} \\
    \quad\quad \texttt{// Usar estadísticas precomputadas o simplificadas} \\
    \quad\quad \texttt{$\tilde{m}_{[\phi_{\ell}]},\,\tilde{C}_{[\phi_{\ell}]} \leftarrow$ estimación acelerada} \\
    \quad\quad \texttt{$w \leftarrow \tilde{C}_{[\phi_{\ell}]}^{-1} \cdot h_{[\phi_{\ell}]}$} \\
    \quad\quad \texttt{$a \leftarrow a + w^{\top} \cdot h_{[\phi_{\ell}]}$} \\
    \quad\quad \texttt{$b \leftarrow b + w^{\top} \cdot (\,r[\phi_{\ell},\ell] - \tilde{m}_{[\phi_{\ell}]}\,)$} \\
    \quad \texttt{end for} \\
    \quad \texttt{S/N($\phi_0$) $\leftarrow$ b / $\sqrt{a}$ (aproximado)} \\
}
\end{algorithm}
El algoritmo \ref{alg:paco2} optimiza el cálculo del S/N mediante aproximaciones computacionales. Sustituye las estimaciones locales exactas de media y covarianza ($\hat{m}$, $\hat{C}$) por versiones precomputadas o simplificadas ($\tilde{m}$, $\tilde{C}$). Mantiene la estructura del Algoritmo 1 pero reduce drásticamente el costo computacional, ideal para procesar grandes volúmenes de datos astronómicos. La relación S/N aproximada ($b/\sqrt{a}$) preserva suficiente precisión para detecciones iniciales.
\newpage
\begin{algorithm}
\caption{PACO estimation – Estimación no sesgada del flujo $\widehat\alpha$}
\label{alg:paco3}
\Begin{
    \texttt{Input: posición $\phi_0$, datos $r[\theta_k,\ell]$, $\widehat\alpha=0$ por defecto, precisión $\epsilon$} \\
    \texttt{$\widehat\alpha_{\text{old}} \leftarrow +\infty$} \\
    \texttt{while $|\widehat\alpha - \widehat\alpha_{\text{old}}| > \epsilon\,\widehat\alpha$:} \\
    \quad \texttt{$\widehat\alpha_{\text{old}} \leftarrow \widehat\alpha$} \\
    \quad \texttt{a ← 0; b ← 0} \\
    \quad \texttt{for $\ell = 1$ to $T$:} \\
    \quad\quad \texttt{// Paso 1: construir colección de parches descontaminados} \\
    \quad\quad \texttt{$\phi_{\ell} = \mathcal{F}_{\ell}(\phi_0)$} \\
    \quad\quad \texttt{$\mathcal{P} \leftarrow \{r[\phi_{\ell'},\ell'] - \widehat\alpha\cdot h[\phi_{\ell'}]\}_{\ell'=1..T}$} \\
    \quad\quad \texttt{// Paso 2: aprender estadísticas con la señal descontaminada} \\
    \quad\quad \texttt{$\hat m(\alpha),\,\hat C(\alpha) \leftarrow$ Eq. (13)} \\
    \quad\quad \texttt{// Paso 3: actualizar términos de estimación} \\
    \quad\quad \texttt{$w \leftarrow \hat C^{-1}(\alpha)\cdot h$} \\
    \quad\quad \texttt{$a \leftarrow a + w^{\top} h$} \\
    \quad\quad \texttt{$b \leftarrow b + w^{\top}(r - \hat m(\alpha))$} \\
    \quad \texttt{end for} \\
    \quad \texttt{$\widehat\alpha \leftarrow \max(b,0) / a$ \quad (Eq. (14))} \\
    \texttt{end while} \\
    \texttt{Output: $\widehat\alpha$ estimado no sesgado} \\
}
\end{algorithm}
El Algoritmo \ref{alg:paco3} estima iterativamente el flujo $\widehat{\alpha}$ de una fuente detectada, corrigiendo el sesgo inducido por la contaminación del fondo. En cada iteración: (1) descuenta la señal estimada ($\widehat{\alpha}h$) de los parches, (2) recalcula las estadísticas del fondo ($\hat{m}(\alpha)$, $\hat{C}(\alpha)$) y (3) actualiza $\widehat{\alpha}$ mediante $\max(b,0)/a$ (Ec. 14). El proceso itera hasta convergencia (precisión $\epsilon$), garantizando estimaciones no sesgadas incluso con ruido no estacionario.


\subsection{Regime Switching Model}

El Regime Switching Model (RSM) constituye nuestra solución al problema de clasificación píxel a píxel en condiciones de incertidumbre extrema (SNR < 3). Este modelo probabilístico jerárquico considera que cada píxel puede estar en dos regímenes: ruido o planeta, con probabilidades $\pi$ y 1-$\pi$ respectivamente \cite{Dahlqvist_2020}.\\
RSM ha demostrado un rendimiento superior en el espacio de la característica operativa del receptor (ROC) en comparación con los mapas de relación señal-ruido estándar generados por algoritmos de post-procesamiento ADI de última generación \cite{Dahlqvist_2020}. Asimismo, se ha confirmado su rendimiento superior en separaciones angulares pequeñas en comparación con los mapas de S/N estándar \cite{Dahlqvist_2021}\\

 La inferencia clásica mediante MCMC resulta prohibitiva para imágenes completas (complejidad O(n³)), por lo que se empleará Inferencia Variacional (VI), que aproxima la posterior mediante distribuciones gaussianas factorizadas optimizadas con gradiente descendente. Se planea utilizar este esquema en PyTorch/Pyro --librerías de python-- donde se define un guide AutoDiagonalNormal que parametriza las distribuciones aproximadas, y optimizamos los parámetros variacionales mediante Adam \cite{kingma2017adammethodstochasticoptimization} sobre el ELBO (Evidence Lower Bound). Esta aproximación reduce la complejidad a O(n) mediante las Ecuaciones \ref{eqrsm1} y \ref{eqrsm2}, permitiendo procesar imágenes completas en minutos en lugar de días. Un aspecto clave es la inicialización inteligente de los parámetros basada en estadísticas locales de los residuos de PACO, lo que acelera la convergencia del algoritmo.\\\\
Modelo jerárquico:
\begin{equation}\label{eqrsm1}
 \begin{aligned}
x_i &\sim \pi \mathcal{N}(\mu_p, \sigma_p) + (1-\pi) \mathcal{N}(0, \sigma_n) \\
\pi &\sim \text{Beta}(1, 1000) \\
\mu_p &\sim \mathcal{N}^+(0, 10)
\end{aligned}
\end{equation}
VI: Aproxima posterior con distribución gaussiana factorizada
donde:
\begin{equation}\label{eqrsm2}
q(\theta) = \prod_i \mathcal{N}(\mu_i, \sigma_i)
\end{equation}
La ventaja computacional es que reemplaza MCMC (O(n³)) por optimización estocástica (O(n)), acelerando el cálculo en GPU con PyTorch.\\
A continuación, se presenta un pseudocódigo conceptual del Algoritmo \ref{alg:rsm_pyro} que ilustra cómo se podría implementar el modelo RSM utilizando Pyro en Python.\\
\begin{algorithm}[H]
\caption{Pseudocódigo Conceptual: Implementación RSM con Pyro}
\label{alg:rsm_pyro}
\SetKwInOut{Input}{Entrada}
\SetKwInOut{Output}{Salida}
\Input{Cubo de imágenes residuales post-PACO, parámetros del modelo}
\Output{Clasificación probabilística píxel a píxel y mapas de incertidumbre}


\textbf{Configuración del modelo bayesiano} \\

\texttt{pyro.clear\_param\_store()}\; 
\texttt{model = RSMModel(n\_pixels=data.shape[-1], prior\_planet\_prob=1e-4)}\; 

\textbf{Distribución guía con inicialización inteligente} \\
\Indp
\texttt{guide = AutoDiagonalNormal(model, init\_loc\_fn=init\_to\_mean)}\; 
\texttt{adam\_params = \{"lr": 0.01, "betas": (0.9, 0.999), ``eps'': 1e-8\}}\; 
\texttt{scheduler = pyro.optim.ExponentialLR(\{"optimizer": torch.optim.Adam, **adam\_params\})}\; 

\textbf{Configuración SVI con early stopping} \\

\texttt{svi = SVI(model, guide, scheduler, loss=Trace\_ELBO())}\; 
\texttt{losses = []}\; 
\For{\texttt{epoch in range(max\_epochs)}}{
    \texttt{loss = svi.step(data)}\; 
    \texttt{losses.append(loss)}\; 
    \If{\texttt{epoch \% 100 == 0}}{
        \texttt{print(f"Epoch \{epoch\}: Loss = \{loss:.4f\}")}\; 
    }
    \textbf{Criterio de convergencia} \\
    \If{\texttt{len(losses) > 50 and np.std(losses[-50:]) < convergence\_tol}}{
        \texttt{break}\; 
    }
}

\textbf{Inferencia posterior y generación de mapas} \\

\texttt{posterior\_samples = guide.sample(n\_samples=1000)}\; 
\texttt{planet\_prob\_map = posterior\_samples['planet\_indicator'].mean(dim=0)}\; 

\end{algorithm}

\subsubsection{Explicación del pseudocodigo}

El pseudocódigo del Algoritmo~\ref{alg:rsm_pyro} ilustra la implementación esquematica del modelo RSM utilizando inferencia variacional \cite{Blei_2017}, reemplazando métodos MCMC tradicionales por una aproximación eficiente. Cabe considerar que esto no ha sido probado ni está completo aún, ha sido creado a modo de referencia de la implementación final en base a la revisión de la literatura. La implementación se estructurará en cuatro etapas clave:

\subsubsection{Configuración conceptual}
\begin{itemize}
    \item \texttt{pyro.clear\_param\_store()}: Reinicia el almacén de parámetros para evitar contaminación entre ejecuciones.
    \item \texttt{RSMModel(n\_pixels=data.shape[-1], prior\_planet\_prob=1e-4)}: 
    \begin{itemize}
        \item \texttt{n\_pixels}: Dimensión espacial de los datos (e.g., $256 \times 256$ para imágenes típicas)
        \item \texttt{prior\_planet\_prob}: Probabilidad a priori $\mathbb{P}(H_1) = 10^{-4}$, donde:
        \begin{equation}
            H_1: \text{El píxel contiene señal planetaria}
        \end{equation}
    \end{itemize}
\end{itemize}

\subsubsection{Optimización Automatizada}
\begin{itemize}
    \item \texttt{AutoDiagonalNormal(model, init\_loc\_fn=init\_to\_mean)}: 
    \begin{itemize}
        \item Aproxima la posterior con distribución normal diagonal
        \item \texttt{init\_to\_mean}: Inicialización inteligente basada en momentos
    \end{itemize}
    \item Configuración de Adam con parámetros:
    \begin{equation}
        \eta = 0.01,\quad \beta_1 = 0.9,\quad \beta_2 = 0.999,\quad \epsilon = 10^{-8}
    \end{equation}
    \item \texttt{ExponentialLR}: Adapta dinámicamente la tasa de aprendizaje $\eta_t = \eta_0 \gamma^t$
\end{itemize}

\subsubsection{Entrenamiento con SVI}
El proceso de entrenamiento minimiza la cota inferior de evidencia (ELBO) basado en \cite{JMLR:v14:hoffman13a}:
\begin{equation}
    \mathcal{L}(\phi) = \mathbb{E}_{q_\phi}[\log p(x,z) - \log q_\phi(z)]
\end{equation}

\begin{itemize}
    \item Implementa early stopping cuando:
    \begin{equation}
        \sigma(\{\mathcal{L}_{t-50}, \dots, \mathcal{L}_t\}) < \tau \quad (\tau = \text{tolerancia})
    \end{equation}
    \item Monitorea convergencia cada 100 épocas
\end{itemize}

\subsubsection{Generación de Mapas}
\begin{itemize}
    \item Muestreo posterior:
    \begin{equation}
        \{\theta^{(s)}\}_{s=1}^{1000} \sim q_\phi(\theta|\mathbf{x})
    \end{equation}
    \item Mapa de probabilidades:
    \begin{equation}
        \hat{p}_{ij} = \frac{1}{1000}\sum_{s=1}^{1000} \mathbb{I}(\theta^{(s)}_{ij} > 0)
    \end{equation}
    donde $(i,j)$ indexan posiciones pixelares.
\end{itemize}

Esta implementación optimizará el balance precisión-computación mediante:
\begin{itemize}
    \item Paralelización automática vía PyTorch
    \item Aproximación variacional en lugar de MCMC costoso
    \item Criterios de convergencia adaptativos
\end{itemize}

\section{Aceleración GPU y Optimizaciones de Memoria}
La implementación eficiente en GPU requiere un entendimiento de la jerarquía de memoria y patrones de acceso optimizados para maximizar el throughput computacional. En la arquitectura CUDA, la memoria se organiza en múltiples niveles con diferentes características de latencia y ancho de banda: memoria global (alta latencia, $\approx$ 400 $\approx$ 600 ciclos), memoria compartida (baja latencia, ~1-2 ciclos), registros (latencia mínima), y memoria constante (solo lectura, cacheable).

Para la implementación de PACO, las optimizaciones se centrarán en minimizar los accesos a memoria global mediante el uso estratégico de memoria compartida. Los kernels CUDA para covarianza local cargan bloques de 32×32 píxeles (1024 elementos) en memoria compartida por streaming multiprocessor, permitiendo que los 32 threads del warp accedan a datos locales con latencia de menores ciclos de memoria global. Esta optimización reduciría el tiempo de acceso a memoria.

La precisión mixta (FP16/FP32) reduce los requerimientos de memoria global en 50\% para operaciones no críticas, mientras mantiene FP32 para acumulaciones sensibles numéricamente como las sumas de covarianza. El kernel fusion combina operaciones consecutivas (carga de datos + normalización + cálculo PCA) en un solo lanzamiento, reduciendo la sobrecarga de sincronización por bloque de operaciones.

Para RSM-VI, se aprovecharán los tensor cores de las GPUs Ampere/Hopper mediante el formato tf32 (tensor float-32), que proporciona el rango dinámico de FP32 con el throughput de FP16, logrando hasta 156 TFLOPS en operaciones matriciales complejas comparado con 19.5 TFLOPS de FP32 convencional. La implementación optimizada permite procesar imágenes de 4096×4096 píxeles, reduciendo el tiempo total de análisis de surveys completos.
\subsection{Paralelización CPU/GPU}
La aceleración mediante arquitecturas heterogéneas se fundamenta en principios como la Ley de Amdahl \cite{10.1145/1465482.1465560}, que proporciona la base teórica para el speedup máximo alcanzable, demostrando que incluso con paralelización perfecta, la porción secuencial del código limita la ganancia potencial. Este principio fundamental se expresa matemáticamente como:

\begin{equation}
S_{\text{max}} = \frac{1}{(1 - p) + \frac{p}{n}}
\end{equation}

donde p representa la fracción paralelizable del código y n el número de unidades de procesamiento. En la práctica, este modelo se complementa con la Ley de Gustafson \cite{gustafson1988}, que introduce el concepto de escalabilidad débil, permitiendo que problemas más grandes puedan lograr mejor escalamiento paralelo. El modelo de ejecución SIMT (Single Instruction Multiple Thread), característico de las GPUs modernas, implementa una variante del paradigma SIMD donde warps de 32 threads ejecutan instrucciones en lock-step \cite{nvidia2023cuda}, con mecanismos para manejar divergencia de control mediante predicación de instrucciones.

Arquitectónicamente, se observa un claro trade-off entre CPUs, optimizadas para baja latencia mediante estructuras complejas como predicción de saltos avanzada, ejecución fuera de orden y pipelines profundos, y GPUs, diseñadas específicamente para alto throughput mediante arrays masivos de núcleos simples (CUDA cores) y jerarquías de memoria especializadas. Las GPUs NVIDIA Ampere representan el estado del arte actual \cite{nvidia2020ampere}, implementando:
\begin{enumerate}
    \item 108 Streaming Multiprocessors organizados en grupos de procesamiento

    \item Subsistema de memoria HBM2e con ancho de banda de 1.5TB/s y baja latencia
    
    \item Cache L2 unificada de 40MB con políticas de reemplazo optimizadas
    
    \item Mecanismos avanzados de prefetching y coalescing de accesos a memoria
\end{enumerate}

\subsection{TensorCores}
Los Tensor Cores representan una evolución de arquitectura en la computación acelerada, implementando operaciones de álgebra lineal en hardware con eficiencia. Su funcionamiento se basa en la operación fundamental de multiplicación-acumulación matricial de la ecuacion \ref{tensorcores}:

\begin{equation}
\label{tensorcores}
D_{m,n} = \sum_{k=1}^{K} A_{m,k} \times B_{k,n} + C_{m,n}
\end{equation}

con soporte para precisiones mixtas (FP16 para los operandos A y B, acumulando en FP32 para C y D). Esta implementación permite alcanzar throughputs teóricos que superan en órdenes de magnitud a las unidades FP32 tradicionales \cite{markidis2018}. La arquitectura Ampere introduce varias mejoras clave:

-Throughput teórico de 312 TFLOPS (FP16) mediante ejecución pipelineada

-Soporte para sparsity 2:4 

-Operaciones MMA (Matrix Multiply-Accumulate) completadas en solo 4 ciclos de reloj

Unidades dedicadas para transformaciones de formato on-the-fly

\subsection{Tecnología Ampere}
La microarquitectura Ampere introduce innovaciones en la gestión de memoria y conectividad, manejando los cuellos de botella tradicionales en aplicaciones de computación intensiva \cite{nvidia2020ampere}. El modelo teórico de ancho de banda efectivo incorpora múltiples factores como se ve en la ecuacion \ref{ampere}:

\begin{equation}
\label{ampere}
\text{Bandwidth}{\text{effective}} = (\text{Bandwidth}{\text{peak}} \times \text{CacheHitRate}) + (\text{CompressionRatio} \times \text{Bandwidth}_{\text{physical}})
\end{equation}
\\
Las implementaciones concretas incluyen:\\
\begin{enumerate}
    \item Subsistema de memoria unificada con acceso CPU-GPU a 100GB/s mediante DMA optimizado
      \item Interconexión NVLink 3.0 con 600GB/s de bandwidth y protocolos mejorados
       \item Tecnología MIG (Multi-Instance GPU) que permite particionamiento físico de recursos
       \item Compresión de datos en tiempo real con ratios de hasta 4:1 para ciertos patrones de acceso
\end{enumerate}




\subsection{CUDA}
CUDA (Compute Unified Device Architecture) \cite{nvidia2023cuda} es un modelo de programación paralela desarrollado por NVIDIA que permite ejecutar código general en GPUs mediante kernels. Estos kernels se ejecutan en una jerarquía de hilos organizados en bloques de 32 hilos \cite{articulo} y grids, donde cada hilo procesa datos de forma concurrente. El modelo se basa en:

\subsubsection{Modelo de Memoria}
La jerarquía de memoria sigue en la ecuacion \ref{memoria}:

\begin{equation}
\label{memoria}
    t_{\text{access}} = 
    \begin{cases}
        1-3\ \text{cycles} & \text{Registros} \\
        20-30\ \text{cycles} & \text{Shared Memory} \\
        200+\ \text{cycles} & \text{Global Memory}
    \end{cases}
\end{equation}

El ancho de banda efectivo se calcula en la ecuacion \ref{coal}:

\begin{equation}
\label{coal}
    \text{BW}_{\text{eff}} = \text{BW}_{\text{peak}} \times \text{Coalescing\%} \times (1 - \text{Bank Conflicts})
\end{equation}

\subsubsection{Ejecución de Kernels}
La configuración de lanzamiento se especifica por \cite{sanders2010} en la ecuación \ref{kernel}:

\begin{equation}
\label{kernel}
    \texttt{kernel<<<gridDim, blockDim, sharedMemSize, stream>>>}
\end{equation}

donde:
\begin{itemize}
    \item \texttt{gridDim}: Dimensión de la malla de bloques
    \item \texttt{blockDim}: Hilos por bloque (máx. 1024)
\end{itemize}

El mapeo de hilos a datos sigue en la ecuación \ref{global}:

\begin{equation}
\label{global}
    \text{GlobalId} = \text{blockIdx} \times \text{blockDim} + \text{threadIdx}
\end{equation}

\subsubsection{Optimización}
La eficiencia se mide mediante la ecuación \ref{flops}:

\begin{equation}
\label{flops}
    \eta = \frac{\text{FLOPs}_{\text{achieved}}}{\text{FLOPs}_{\text{theoretical}}}
\end{equation}

Factores críticos:
\begin{itemize}
    \item Coalescing de accesos a memoria global
    \item Minimización de divergencia de warps
    \item Uso de memoria compartida como caché programable
\end{itemize}
\section{Validación del Modelo}
En esta sección se presentan las métricas con las cuales se evaluará el pipeline y se verificará si el resultado de la investigación fue positivo o negativo.
\subsection{Métricas de Desempeño}
Las métricas estándar de detección para el pipeline híbrido RSM-PACO se definen mediante la Ecuación \ref{eq:fdr_mdr}:

\begin{equation}\label{eq:fdr_mdr}
\text{FDR}  = \frac{\text{FP}}{\text{TP} + \text{FP}}, \quad 
\text{MDR} = \frac{\text{FN}}{\text{TP} + \text{FN}}
\end{equation}

donde FDR (False Discovery Rate) representa la tasa de falsos positivos y MDR (Missed Detection Rate) indica la tasa de detecciones perdidas. Para evaluar el desempeño del pipeline híbrido RSM-PACO, se prioriza minimizar el MDR acercando el valor lo más posible a cero, ya que se busca maximizar la capacidad de detectar planetas reales sin perder señales genuinas por clasificaciones incorrectas como ruido residual.


\begin{table}[H]
\centering
\caption{Relaciones entre métricas de evaluación para detección de exoplanetas}
\label{tab:metricas_avanzadas}
\begin{tabular}{llll}
\toprule
\textbf{Métrica} & \textbf{Fórmula} & \textbf{Relación con FDR/MDR} & \textbf{Interpretación} \\
\midrule
Precisión & $\frac{\text{TP}}{\text{TP} + \text{FP}}$ & $\text{Precisión} = 1 - \text{FDR}$ & Exactitud de las detecciones \\[0.2cm]
FDR & $\frac{\text{FP}}{\text{TP} + \text{FP}}$ & $\text{FDR} = 1 - \text{Precisión}$ & Contaminación por falsos positivos \\[0.2cm]
Recall & $\frac{\text{TP}}{\text{TP} + \text{FN}}$ & $\text{Recall} = 1 - \text{MDR}$ & Capacidad de detectar planetas reales \\[0.2cm]
MDR & $\frac{\text{FN}}{\text{TP} + \text{FN}}$ & $\text{MDR} = 1 - \text{Recall}$ & Planetas reales no detectados \\[0.2cm]
F1-Score & $\frac{2 \cdot \text{Prec} \cdot \text{Rec}}{\text{Prec} + \text{Rec}}$ & Combina FDR y MDR & Balance entre precisión y cobertura \\
\bottomrule
\end{tabular}
\end{table}
donde:\\
TP = True positive\\
FP = False positive\\






\chapter{Estado del Arte} %* estado del arte es la etapa del proceso de investigación posterior a la elección y justificación del tema. Una vez que hayas escogido su tema de interés, sigue la parte en la que se tendrá que leer lo que se ha escrito en torno a esa temática.\\
%* Estado del arte es: Estudio analítico del conocimiento acumulado que hace parte de la investigación documental y que tiene como objetivo inventariar y sistematizar la producción en un área del conocimiento de los últimos años”
%Introducción al capítulo

El estado del arte constituye una revisión sistemática del conocimiento acumulado en el campo de la detección de exoplanetas mediante técnicas de aprendizaje estadístico y procesamiento de imágenes de alto contraste. Este capítulo sintetiza los avances recientes en:

\begin{enumerate}
    \item Métodos Tradicionales de Sustracción de PSF y su Evolución
    \item Avances en Modelado de Ruido y Detección Estadística
    \item Brechas tecnológicas que justifican el desarrollo de un enfoque híbrido RSM-PACO para observación directa.
\end{enumerate}


\section{Métodos tradicionales} Se abordan técnicas como PCA, sustracción PSF y LOCI para dar contexto de cuáles se utilizan y cómo deben compararse a la solución propuesta.
\subsection{PCA} 
Es una técnica de reducción de dimensionalidad ampliamente adoptada en la imagen de alto contraste para la supresión de speckles. En esencia, PCA transforma datos de alta dimensión en un espacio de menor dimensión al identificar direcciones ortogonales (componentes principales) que capturan la máxima varianza en el conjunto de datos. En el contexto de las imágenes astronómicas, esto implica realizar una Descomposición de Valores Singulares (SVD) en el cubo de imágenes para identificar los patrones de ruido dominantes, que luego se restan. \\

A pesar de su eficiencia computacional y uso generalizado, una limitación significativa de PCA en la detección de exoplanetas es su propensión a la ``auto-sustracción''. Esto ocurre porque PCA puede aprender inadvertidamente la forma de la PSF del telescopio, que, para un alto número de componentes de PCA, captura no solo el ruido estructurado de los speckles sino también la débil y característica forma de la propia señal del planeta. Esto conduce a una pérdida de sensibilidad, particularmente en regiones cercanas a la estrella (por ejemplo, por debajo de 0.3 arcosegundos), donde la señal del planeta es más débil y más susceptible a la sobre-sustracción. En consecuencia, la dependencia de PCA del número de componentes a menudo requiere un ajuste manual, lo que complica la optimización de hiperparámetros. La limitación inherente de PCA de la ``auto-sustracción'' revela una compensación fundamental y persistente en la supresión de speckles: cuanto más agresivamente un algoritmo intenta modelar y eliminar el abrumador ruido estelar, mayor es el riesgo de atenuar o eliminar simultáneamente la débil y deseada señal astrofísica. Esta situación subraya la necesidad crítica de modelos de ruido más sofisticados que puedan distinguir robustamente entre el ruido y las señales verdaderas, en lugar de simplemente identificar patrones dominantes. Esto impulsa el desarrollo de métodos que incorporan priors físicos o una discriminación estadística más matizada, buscando superar esta dicotomía intrínseca entre la supresión de ruido y la preservación de la señal.\\

\subsection{Combinación de Imágenes Optimizada Localmente (LOCI)}
El algoritmo LOCI, introducido por \citep{Lafreniere_2007} en 2007, representa un avance en las técnicas de sustracción de PSF. LOCI opera construyendo una imagen de PSF de referencia optimizada a partir de un conjunto de imágenes de referencia disponibles. Esta imagen de referencia se construye como una combinación lineal de imágenes seleccionadas, donde los coeficientes de la combinación se optimizan de forma independiente dentro de múltiples subsecciones localizadas de la imagen para minimizar el ruido residual. Esta estrategia de optimización local permite a LOCI adaptarse a la naturaleza espacialmente variable del ruido de los speckles.  \\

Aunque es altamente eficaz en la supresión de speckles y en el logro de un alto contraste, los métodos basados en LOCI son conocidos por introducir sesgos en la astrometría (posición) y la fotometría (brillo) de las fuentes débiles detectadas. Esto se debe a menudo a la inevitable auto-sustracción parcial de la señal del planeta durante el proceso de modelado del ruido, lo que puede distorsionar las verdaderas características del exoplaneta. \\

La fortaleza de LOCI radica en su optimización local, que es muy adecuada para manejar la naturaleza no estacionaria del ruido de los speckles. Sin embargo, este enfoque localizado, cuando se implementa sin suficiente contexto global o restricciones físicas, puede conducir inadvertidamente a inconsistencias en toda la imagen o a sesgos en los parámetros astrofísicos derivados. Esto pone de manifiesto un desafío crítico en el procesamiento de imágenes: equilibrar la necesidad de adaptabilidad local al ruido complejo con el requisito de una recuperación de la señal globalmente consistente y físicamente precisa. Si las optimizaciones locales son demasiado agresivas o no están limitadas por información global (por ejemplo, la forma esperada de una PSF o la variación suave del fondo a través del campo), pueden introducir artefactos o distorsionar las señales de maneras que son localmente óptimas para la sustracción de ruido, pero perjudiciales para la interpretación científica general.\newpage
\section{ Avances en Modelado de Ruido y Detección Estadística} Se abordan los algoritmos/métodos más utilizados actualmente.
\subsection{PACO}
PACO es un algoritmo híbrido que se sitúa en el núcleo de los pipelines de reducción de ruido, combinando la descomposición matricial con el modelado estadístico espacial.\\

PACO se distingue por emplear un enfoque de aprendizaje estadístico para modelar las fluctuaciones de fondo, incorporando covarianzas de parches no estacionarios, lo que le permite aprender un modelo gaussiano multivariante no estacionario del entorno. Esto es crucial para una estimación precisa del fondo y lo diferencia de muchos métodos anteriores que se basaban en restas o transformaciones más sencillas. PACO es fotométricamente insesgado y permite evaluar directamente la tasa de falsas alarmas sin simulaciones. La implementación de este proceso dual en CUDA, utilizando cuBLAS para la SVD y kernels personalizados para la covarianza, busca una aceleración de 50x respecto a las versiones de CPU. \\

A pesar de sus ventajas, PACO presenta limitaciones. Aunque es eficiente y puede ejecutarse en GPU, sufre una pérdida de sensibilidad en zonas cercanas a la estrella  debido a un suavizado excesivo que elimina tanto el ruido como las señales débiles. Además, requiere un ajuste manual del número de componentes de PCA, lo que complica la optimización de hiperparámetros. La investigación sobre PACO ha continuado, con trabajos como \citep{Flasseur_2020a} (2020), que describe PACO ASDI, un algoritmo para la detección y caracterización de exoplanetas en imagen directa con espectrógrafos de campo integral, y \citep{Flasseur_2020b} (2020), que aborda la robustez frente a fotogramas defectuosos. El trabajo original de PACO fue publicado por \citep{Flasseur_2018} en 2018.



\subsection{RSM}
El Modelo de Cambio de Régimen (RSM) es una solución para el problema de clasificación píxel a píxel en condiciones de incertidumbre extrema (SNR < 3). Este modelo probabilístico jerárquico considera que cada píxel puede estar en dos regímenes: ruido o planeta, con probabilidades $\pi$ y 1-$\pi$ respectivamente. El RSM modela explícitamente la probabilidad de que un píxel dado contenga solo ruido residual o una señal planetaria genuina, ofreciendo ventajas teóricas como la cuantificación natural de la incertidumbre y una mayor robustez frente a falsos positivos. Sin embargo, es computacionalmente demasiado complejo debido al usom de MCMC. Para superar esto, se propone el uso de Inferencia Variacional, que aproxima la posterior mediante distribuciones gaussianas factorizadas optimizadas con gradiente descendente. Esta aproximación reduce la complejidad a O(n), permitiendo procesar imágenes completas en minutos en lugar de días. Un aspecto clave es la inicialización inteligente de los parámetros basada en estadísticas locales de los residuos de PACO, lo que acelera la convergencia del algoritmo. El trabajo seminal que introdujo el RSM para la detección de exoplanetas fue realizado por \citep{Dahlqvist_2020} en 2020, con mejoras posteriores en \citep{Dahlqvist_2021} en 2021.

\subsection{Otros Métodos Recientes, Paquetes de Software para Procesamiento de Imágenes de Alto Contraste y Brechas Actuales}

VIP (Vortex Image Processing): VIP es un paquete de Python de código abierto dedicado a la imagen de alto contraste astronómica, desarrollado para proporcionar un marco flexible para el procesamiento de datos e imágenes \citep{Gomez_Gonzalez_2017}. Fue iniciado por Carlos Alberto Gomez Gonzalez y ha sido mantenido y desarrollado por el Dr. Valentin Christiaens desde 2020. VIP implementa funcionalidades para construir pipelines de procesamiento de datos de alto contraste, abarcando algoritmos de pre- y post-procesamiento, estimación de posición y flujo de fuentes potenciales, y generación de curvas de sensibilidad \citep{Gomez_Gonzalez_2017}. Entre las técnicas de sustracción de PSF de referencia para el post-procesamiento ADI, VIP incluye varias variantes de algoritmos basados en PCA (como PCA anular e incremental) y una implementación de NMF \citep{Gomez_Gonzalez_2017}. VIP también ofrece la posibilidad de realizar SVD en GPU utilizando CuPy o PyTorch, y optimizar rutinas de corrección de píxeles defectuosos con Numba, lo que puede ofrecer mejoras de velocidad significativas (hasta 50x para rotaciones de imagen con OpenCV) \\


PynPoint, un paquete de código abierto en Python, ofrece un enfoque modular basado en PCA y análisis estadístico independiente, optimizado para datos de SPHERE/VLT y JWST. Sus ventajas incluyen la integración modular de múltiples algoritmos (PCA, NMF, sustracción de PSF) y precisión fotométrica con correcciones de sesgo y estimación de incertidumbres. Sin embargo, PynPoint no modela el ruido no estacionario como PACO ni segmenta regímenes como RSM, y su dependencia de PCA limita la sensibilidad en contrastes bajos (<1e-6). El trabajo principal de PynPoint fue publicado por \citep{Stolker_2019} en 2019.La Factorización de Matrices No Negativas (NMF) es otra técnica iterativa aplicada al postprocesamiento de datos de imagen directa para la supresión de speckles. NMF construye una base de componentes no ortogonales y no negativos a partir de imágenes de referencia. Sus ventajas incluyen no requerir una selección previa de referencias y una mejor preservación de la morfología de discos débiles. Aunque computacionalmente más costosa que KLIP para la construcción de componentes, NMF es menos propensa al sobreajuste y mejora la eliminación de ruido. El trabajo de \citep{Ren_2018} en 2018 es un ejemplo clave de la aplicación de NMF en este campo.\\

Los enfoques de aprendizaje profundo (Deep Learning) han emergido como una vía prometedora para mejorar la detección de exoplanetas y la reducción de speckles, a menudo combinando modelos estadísticos con redes neuronales. Ejemplos incluyen deep PACO \citep{Flasseur_2023} (2023), que combina el modelo estadístico de PACO con redes neuronales convolucionales para mejorar la sensibilidad de detección, y torchKLIP \citep{Soummer_2012} (2012), un paquete de PyTorch que busca optimizar la sustracción de PSF. PACOME, una extensión de PACO, se enfoca en la combinación óptima de observaciones de múltiples épocas para la detección conjunta de exoplanetas y la estimación de órbitas. Este algoritmo integra la exploración de órbitas keplerianas dentro de un formalismo de detección estadística, permitiendo la co-adición constructiva de señales débiles y mejorando la sensibilidad de detección con el número de épocas. El trabajo clave para PACOME fue publicado por \citep{Dallant_2023} en 2023.\\

A pesar de estos avances, existe una brecha técnica importante: la necesidad de un marco que preprocese eficientemente (acelerando PACO en GPU) y clasifique señales con modelos bayesianos escalables (como RSM que utiliza inferencia variacional). La dependencia de PCA en PynPoint, por ejemplo, limita la sensibilidad en bajos contrastes, justificando la integración de componentes como PACO (para ruido no estacionario) y RSM (para segmentación temporal). Esto subraya la necesidad de soluciones híbridas que combinen lo mejor de diferentes enfoques para abordar los desafíos persistentes en la detección de exoplanetas de alto contraste. 


\label{cap:estado_del_arte}

\chapter{Desarrollo del Trabajo} \label{cap:metodologia}
%Apoyo
%%7. Desarrollo del Trabajo ( aqui se pueden extender)
%Introduccion al capitulo, va entre el titulo del capitulo y la primera seccion.
%7.1. Diseño de arquitectura
%7.1.1. Configuración Inicial del Entorno
%7.1.2. Arquitectura Implementación Final
%7.2. Estructura del código
%7.2.1. Backend
%7.2.2. Frontend
%7.3. Desarrollo del Software
%7.3.1. Problemas Durante el Desarrollo
%7.3.2. Pruebas, Optimización y Métricas Finales
%7.5. Conclusión del Capítulo





En el presente capítulo se explica el desafío abordado y la solución propuesta en este trabajo. Luego, se describen los algoritmos utilizados como base.Por último, se detalla el desarrollo del pipeline propuesto.

%\section{Planificación de la experimentación}

%\section{Problema a abordar}

	
%\section{Algoritmo utilizado: Método Formas de Contexto}
%\subsubsection{Filtros aplicados}
%\subsection{Capas superficiales}
%\subsection{Viabilidad de una pose}
%\subsection{Puntuación de la pose $\pi$}
%Si la pose evaluada no presenta superposición grande o aguda, es puntuada empleando para ello el cálculo del BSA. Luego, se clasifica junto al resto de formas de contexto.

%\subsection{Datos de Entrada}
%\section{Softwares y lenguajes utilizados}
%\subsection{Plataforma de Desarrollo}
%\subsection{UCSF Chimera}
%\subsection{Programa MSMS}
%\subsection{Blender}
%\subsection{Software VMD}
%section{Lenguajes}
\section{Metodología}
Este trabajo busca abordar la brecha técnica identificada, proponiendo una solución híbrida combinando la eficiencia computacional de PACO y la robustez estadística de RSM que permita preprocesar de forma eficiente las imágenes y clasificar señales planetarias usando modelos bayesianos entrenables a gran escala. Para ello se pretende diseñar un pipeline modular que opera en tres secuenciales:
\begin{enumerate}
    \item Etapa de Preprocesamiento (PACO-GPU)
        \begin{itemize}
        \item Implementación paralelizada en CUDA \cite{7482491}
        \item Kernel optimizado para operaciones de covarianza local
        \item Gestión de memoria mediante paginación GPU-CPU
        \end{itemize}
    \item Etapa de Inferencia (RSM-VI)
        \begin{itemize}
        \item Modelo generativo con capa estocástica
        \item Esquema de entrenamiento semi-supervisado
        \item Regularización basada en física óptica
        \end{itemize}
    \item Etapa de Postprocesamiento   
        \begin{itemize}
        \item Filtrado espacial adaptativo
        \item Clasificación probabilística
        \item Generación de mapas de incertidumbre
        \end{itemize}
\end{enumerate}
La solución propuesta se basa en pilares técnicos bien establecidos como:\\
\begin{enumerate}
    \item Procesamiento paralelo en GPU: Implementaremos PACO utilizando CUDA para acelerar las operaciones matriciales, aprovechando trabajos previos en procesamiento de imágenes astronómicas. Esto reduciría el tiempo de preprocesamiento considerablemente.\\\\
    \item Inferencia variacional (VI): En lugar del tradicional MCMC (Markov Chain Monte Carlo) usado en RSM, aplicaremos técnicas de machine learning bayesiano que permiten entrenamiento distribuido [6]. VI aproxima distribuciones complejas mediante modelos más simples y optimizables por gradiente descendente.
    \item Precisión numérica mixta: Combinaremos cálculos en FP16 y FP32 siguiendo estándares establecidos por NVIDIA , lo que reduce los requerimientos de memoria sin afectar significativamente la calidad de los resultados.\\\\
\end{enumerate}


\subsubsection{Especificaciones Técnicas}
La mayoria de las funcionalidades necesarias se harán utilizando el VIP Package 1.6.6 cuyo repositorio es abierto \citep{VIP_github}, este tambien contiene datos de prueba a utilizar. 
\section{Módulo PACO-GPU} El repositorio de PACO original, tambien abierto en github (no tiene version) que corresponde a \citep{PACO_github}
\begin{enumerate}
    \item Arquitectura: 3 capas de procesamiento paralelo
        \begin{itemize}
        \item Normalización y alineamiento
        \item Descomposición PCA por bloques
        \item Cálculo de covarianza local
        \end{itemize}
    \item Precisión numérica: FP16 para speckles, FP32 para señales    
    \item Throughput: 42 imágenes/min (4096×4096 px)  
\end{enumerate}
\section{Módulo RSM-VI}
\begin{enumerate}
    \item Red bayesiana con 3 capas ocultas     
    \item Función de pérdida: ELBO modificado \cite{sjölund2023tutorialparametricvariationalinference}
    \item Optimizador: AdamW con warmup  
    \item Tasa de aprendizaje: 1e-4 (decaimiento exponencial)    
\end{enumerate}
\section{Flujo de Datos} El pipeline maneja tres streams concurrentes.
\begin{enumerate}
    \item Stream de imágenes crudas (HDF5)    
    \item Stream de parámetros instrumentales (JSON)
    \item Stream de modelos pre-entrenados (ONNX)
\end{enumerate}
\section{Requisitos del Sistema} 
\begin{enumerate}
    \item Hardware: GPU NVIDIA con arquitectura Ampere+  
    \item Software: CUDA 11.7, PyTorch 2.0+
\end{enumerate}

En la tabla \ref{tab:retos_solucion} se resumen los retos informáticos para las operaciones, el enfoque propuesto para cada uno de ellos y las tecnologías a utilizar.

% Segunda tabla
	\begin{table}[H]
		\centering
		\small
		\begin{tabular}{p{5cm}p{5cm}p{5cm}}
			\toprule
			\textbf{Reto Computacional} &  
                \textbf{Enfoque propuesto} &  
                \textbf{Tecnologías Usadas}\\
			\midrule
			\textbf{Operaciones matriciales masivas}  & PACO en GPU     & CUDA, cuBLAS    \\
			\midrule
			\textbf{Inferencia lenta} & RSM con VI & PyTorch, Pyro     \\
            \midrule
            \textbf{Manejo de grandes datasets} & Pipelining eficiente & Dask, Memmap     \\
			\midrule
			\textbf{Precisión numérica} & FP16/FP32 mixtol & NVIDIA Tensor Cores     \\
			\bottomrule
		\end{tabular}
        \caption{Retos Informáticos vs Solución propuesta}
        \label{tab:retos_solucion}
	\end{table}
    \newpage
\subsection{Diagrama de flujo de la solución propuesta}
La Figura \ref{fig:diagrama-solucion} presenta el diagrama de flujo pipeline híbrido propuesto. Este diagrama ilustra las etapas principales del sistema y el flujo de datos entre ellas. Las formas romboidales representan estados intermedios de transformación de datos durante el procesamiento, no condiciones lógicas, sino puntos donde los datos cambian de formato o se integran múltiples flujos. Los colores del diagrama identifican las diferentes etapas: morado para PACO-GPU, rosado para RSM-VI, verde para postprocesamiento, y amarillo para entrada/salida del sistema.

\begin{figure}[H]
\centering
\begin{tikzpicture}[
    node distance=0.8cm and 1.2cm,
    box/.style={rectangle, rounded corners, minimum width=2cm, minimum height=0.6cm, text centered, draw=black, font=\scriptsize},
    paco/.style={box, fill=blue!20},
    rsm/.style={box, fill=purple!20},
    post/.style={box, fill=green!20},
    data/.style={diamond, aspect=2, minimum width=2cm, minimum height=0.6cm, text centered, draw=black, fill=orange!20, font=\scriptsize},
    io/.style={box, fill=yellow!20},
    arrow/.style={->, >=stealth, line width=0.5pt}
]

% Entrada
\node (input) [io] {Entrada: Cubos FITS};

% PACO-GPU - Primera fila
\node (paco1) [paco, below=of input, xshift=-2.5cm] {Corrección píxeles};
\node (paco2) [paco, right=of paco1] {Alineamiento};
\node (paco3) [paco, right=of paco2] {PCA speckles};
\node (paco4) [paco, right=of paco3] {Covarianza local};

% Estado intermedio 1
\node (inter1) [data, below=of paco2.south, xshift=0.6cm] {Imágenes procesadas};

% RSM-VI - Segunda fila
\node (rsm1) [rsm, below=of inter1, xshift=-1.8cm] {Modelo RSM};
\node (rsm2) [rsm, right=of rsm1] {Inferencia VI};
\node (rsm3) [rsm, right=of rsm2] {Optimización};
\node (rsm4) [rsm, right=of rsm3] {Modelo jerárquico};

% Estado intermedio 2
\node (inter2) [data, below=of rsm2.south, xshift=0.6cm] {Mapas probabilidad};

% Postprocesamiento - Tercera fila
\node (post1) [post, below=of inter2, xshift=-1.2cm] {Clasificación};
\node (post2) [post, right=of post1] {Mapas incertidumbre};
\node (post3) [post, right=of post2] {Filtrado espacial};

% Salida
\node (output) [io, below=of post2] {Detecciones finales};

% Conexiones principales
\draw [arrow] (input) -- (paco1);
\draw [arrow] (input) -- (paco2);
\draw [arrow] (input) -- (paco3);
\draw [arrow] (input) -- (paco4);

\draw [arrow] (paco1) -- (inter1);
\draw [arrow] (paco2) -- (inter1);
\draw [arrow] (paco3) -- (inter1);
\draw [arrow] (paco4) -- (inter1);

\draw [arrow] (inter1) -- (rsm1);
\draw [arrow] (inter1) -- (rsm2);
\draw [arrow] (inter1) -- (rsm3);
\draw [arrow] (inter1) -- (rsm4);

\draw [arrow] (rsm1) -- (inter2);
\draw [arrow] (rsm2) -- (inter2);
\draw [arrow] (rsm3) -- (inter2);
\draw [arrow] (rsm4) -- (inter2);

\draw [arrow] (inter2) -- (post1);
\draw [arrow] (inter2) -- (post2);
\draw [arrow] (inter2) -- (post3);

\draw [arrow] (post1) -- (output);
\draw [arrow] (post2) -- (output);
\draw [arrow] (post3) -- (output);

\end{tikzpicture}
\caption{Pipeline híbrido RSM-PACO para detección de exoplanetas. Los colores distinguen las etapas.}
\label{fig:diagrama-solucion}
\end{figure}



%\textcolor{red}{La sección de pruebas y validaciones van en este capitulo}

%\chapter{Resultados}  \label{cap:aplicacion}
%En este capítulo se presentan los resultados obtenidos del análisis de profiling y benchmarking realizado sobre los algoritmos PACO y RSM, así como las optimizaciones propuestas para mejorar su rendimiento computacional. El análisis se realizó utilizando el código original del repositorio PACO-master, que implementa fielmente el Algorithm 1 del trabajo de Flasseur et al. (2018), sin simplificaciones que pudieran afectar la validez de los resultados.

\section{Análisis de Profiling y Cuellos de Botella}

El análisis de profiling constituye una etapa fundamental para identificar las operaciones computacionalmente más costosas dentro de los algoritmos. Este análisis se realizó utilizando herramientas de profiling estándar de Python (cProfile) sobre el código original de PACO-master, ejecutado con diferentes tamaños de datasets sintéticos y datos reales del Exoplanet Imaging Data Challenge.

\subsection{Identificación de Cuellos de Botella en PACO}

El análisis de profiling reveló que la función \texttt{pixelCalc()} constituye el cuello de botella dominante en el algoritmo PACO, consumiendo entre el 93\% y el 98\% del tiempo total de ejecución, dependiendo del tamaño del dataset y el número de frames temporales. Esta función es responsable del cálculo de la media y la matriz de covarianza inversa para cada patch temporal, operaciones que se ejecutan de forma iterativa para cada píxel de la imagen.

Dentro de \texttt{pixelCalc()}, se identificaron tres componentes principales que contribuyen significativamente al tiempo de ejecución:

\begin{enumerate}
    \item \textbf{\texttt{sampleCovariance()}}: Esta función calcula la matriz de covarianza muestral utilizando el método de shrinkage de Ledoit-Wolf, consumiendo entre el 72\% y el 91\% del tiempo total, dependiendo del número de frames. La implementación actual utiliza una list comprehension que ejecuta \texttt{np.cov()} de forma iterativa, lo que impide aprovechar las optimizaciones vectorizadas de NumPy.
    
    \item \textbf{\texttt{np.linalg.inv()}}: La inversión de matrices densas consume entre el 44\% y el 46\% del tiempo total en datasets medianos y grandes. Esta operación es necesaria para calcular la matriz de covarianza inversa $\hat{C}^{-1}_{[\phi_\ell]}$ utilizada en el cálculo de los pesos óptimos $w$.
    
    \item \textbf{\texttt{shrinkageFactor()}}: El cálculo del factor de regularización de Ledoit-Wolf representa entre el 3\% y el 10\% del tiempo total, siendo más significativo en datasets pequeños donde el overhead relativo es mayor.
\end{enumerate}

La Tabla \ref{tab:cuellos_botella_paco} presenta un desglose detallado del tiempo de ejecución por función para diferentes configuraciones de datasets.

\begin{table}[H]
    \centering
    \small
    \begin{tabular}{lcccc}
        \toprule
        \textbf{Función} & \textbf{Small} & \textbf{Medium} & \textbf{NACO} & \textbf{SPHERE} \\
        & \textbf{(10 frames)} & \textbf{(20 frames)} & \textbf{(61 frames)} & \textbf{(24 frames)} \\
        \midrule
        \texttt{pixelCalc()} & 93.4\% & 94.4\% & 96.6\% & 94.0\% \\
        \texttt{sampleCovariance()} & 71.9\% & 81.5\% & 91.0\% & 82.6\% \\
        \texttt{<listcomp>} & 68.0\% & 78.2\% & 82.8\% & 79.4\% \\
        \texttt{shrinkageFactor()} & 10.1\% & 6.0\% & 2.7\% & 5.3\% \\
        \texttt{getPatch()} & 3.9\% & 4.1\% & 2.8\% & 4.5\% \\
        \midrule
        \textbf{Tiempo Total} & 0.180s & 1.220s & 10.798s & 1.677s \\
        \bottomrule
    \end{tabular}
    \caption{Distribución del tiempo de ejecución por función en PACO para diferentes configuraciones de datasets. Los porcentajes se calculan sobre el tiempo total de ejecución.}
    \label{tab:cuellos_botella_paco}
\end{table}

\subsection{Identificación de Cuellos de Botella en RSM}

Para el algoritmo RSM, el análisis de profiling reveló que la función \texttt{GaussianMixture.fit()} domina extremadamente el tiempo de ejecución, consumiendo aproximadamente el 98\% del tiempo total. Esta función implementa el algoritmo de Expectation-Maximization (EM) utilizado para estimar los parámetros del modelo de mezcla gaussiana que caracteriza los dos regímenes (ruido y planeta).

Dentro de \texttt{GaussianMixture.fit()}, las iteraciones del algoritmo EM representan aproximadamente el 95\% del tiempo, mientras que la verificación de convergencia y la inicialización de parámetros consumen el 2\% y el 1\% restante, respectivamente. Esta distribución del tiempo sugiere que existe un potencial significativo para optimización mediante la implementación de criterios de early stopping más agresivos, que podrían reducir el número de iteraciones necesarias sin comprometer la calidad de la convergencia.

\section{Análisis de Escalabilidad}

El análisis de escalabilidad permite comprender cómo el rendimiento del algoritmo se comporta al incrementar el tamaño de los datos de entrada. Para este análisis, se ejecutaron benchmarks con tres configuraciones de datasets sintéticos: Small (10 frames, 32×32 píxeles), Medium (20 frames, 64×64 píxeles) y Large (30 frames, 96×96 píxeles).

\subsection{Escalabilidad de PACO}

Los resultados del benchmarking revelan que PACO exhibe una complejidad computacional super-lineal, con un exponente estimado de aproximadamente $O(N^{2.29})$ cuando se considera el número total de píxeles. Esta complejidad se confirma mediante análisis de regresión logarítmica sobre los tiempos de ejecución medidos.

La Tabla \ref{tab:escalabilidad_paco} presenta los resultados detallados del benchmarking para diferentes tamaños de datasets.

\begin{table}[H]
    \centering
    \small
    \begin{tabular}{lcccc}
        \toprule
        \textbf{Dataset} & \textbf{Tiempo} & \textbf{Throughput} & \textbf{Memoria} & \textbf{SNR máximo} \\
        & \textbf{(s)} & \textbf{(px/s)} & \textbf{(MB)} & \\
        \midrule
        Small (10×32×32) & 0.411 & 155.8 & 0.65 & 4.706 \\
        Medium (20×64×64) & 33.392 & 7.7 & 2.62 & 2.953 \\
        Large (30×96×96) & 62.503 & 4.1 & 2.34 & 3.386 \\
        \midrule
        \textbf{Escalabilidad} & \textbf{81.3×} & \textbf{-95\%} & \textbf{3.6×} & - \\
        \textbf{(Small→Medium)} & & & & \\
        \bottomrule
    \end{tabular}
    \caption{Resultados de benchmarking de PACO para diferentes tamaños de datasets sintéticos. El throughput se calcula como píxeles procesados por segundo.}
    \label{tab:escalabilidad_paco}
\end{table}

Se observa una degradación significativa del throughput al incrementar el tamaño del dataset: el throughput disminuye de 155.8 píxeles/segundo en el dataset Small a 4.1 píxeles/segundo en el dataset Large, representando una pérdida del 97.4\% en eficiencia. Esta degradación se atribuye principalmente a la naturaleza cuadrática del cálculo de covarianza, que escala con el cuadrado del número de frames temporales $T$, y a la complejidad cúbica de la inversión de matrices.

\subsection{Escalabilidad con Datos Reales}

Para validar los resultados obtenidos con datos sintéticos, se realizaron benchmarks adicionales utilizando datos reales del Exoplanet Imaging Data Challenge. Se utilizaron dos datasets principales:

\begin{enumerate}
    \item \textbf{NACO Beta Pictoris}: 61 frames de 101×101 píxeles (2.37 MB)
    \item \textbf{SPHERE V471 Tauri}: 24 frames de 132×132 píxeles (1.60 MB)
\end{enumerate}

Los resultados muestran que el tiempo de procesamiento escala de forma aproximadamente lineal con el número de píxeles evaluados, con un coeficiente de determinación $R^2 = 1.0000$ para ambos datasets. Para el dataset NACO Beta Pictoris, el tiempo promedio por píxel es de 197.08 ms, mientras que para SPHERE V471 Tauri es de 31.63 ms, siendo este último 6.2× más rápido debido al menor número de frames (24 vs 61).

Proyectando estos resultados a imágenes completas, se estima que el procesamiento de una imagen completa de NACO Beta Pictoris (622,261 píxeles) requeriría aproximadamente 34.3 horas de tiempo de ejecución, mientras que para SPHERE V471 Tauri (418,176 píxeles) se requerirían aproximadamente 3.7 horas. Estos tiempos confirman la necesidad crítica de optimización para hacer viable el procesamiento de imágenes completas en aplicaciones prácticas.

\section{Análisis de Paralelización}

El análisis de paralelización se enfoca en identificar las oportunidades para distribuir el trabajo computacional entre múltiples unidades de procesamiento, ya sea mediante paralelización a nivel de CPU (threading o multiprocessing) o mediante aceleración en GPU.

\subsection{Paralelizabilidad del Loop Principal de PACO}

El análisis del código fuente de PACO revela que el loop principal en la función \texttt{PACOCalc()} es perfectamente paralelizable, ya que cada iteración del loop sobre las posiciones angulares $\phi_0$ es completamente independiente de las demás. Esta independencia se debe a que el cálculo del S/N para cada posición angular no depende de los resultados de otras posiciones, cumpliendo así con los principios de Bernstein para paralelización.

El Algoritmo \ref{alg:paco_secuencial} presenta la estructura del loop principal secuencial, mientras que el Algoritmo \ref{alg:paco_paralelo} muestra la versión paralelizada propuesta.

\begin{algorithm}[H]
\caption{PACO: Versión Secuencial del Loop Principal}
\label{alg:paco_secuencial}
\Begin{
    \texttt{Input: $\phi_0$s, image\_stack, angles, psf, mask} \\
    \texttt{Output: a, b} \\
    \texttt{// Inicialización} \\
    \texttt{npx ← len($\phi_0$s)} \\
    \texttt{a ← zeros(npx)} \\
    \texttt{b ← zeros(npx)} \\
    \texttt{// Loop principal secuencial} \\
    \texttt{for i ← 0 to npx - 1 do} \\
    \quad \texttt{$\phi_0$ ← $\phi_0$s[i]} \\
    \quad \texttt{angles\_px ← getRotatedPixels($\phi_0$, angles)} \\
    \quad \texttt{for $\ell$ ← 0 to len(angles\_px) - 1 do} \\
    \quad\quad \texttt{patch[$\ell$] ← getPatch(angles\_px[$\ell$], mask)} \\
    \quad\quad \texttt{m[$\ell$], Cinv[$\ell$] ← pixelCalc(patch[$\ell$])} \\
    \quad\quad \texttt{h[$\ell$] ← psf[mask]} \\
    \quad \texttt{end for} \\
    \quad \texttt{a[i] ← al(h, Cinv)} \\
    \quad \texttt{b[i] ← bl(h, Cinv, patch, m)} \\
    \texttt{end for} \\
    \texttt{return a, b}
}
\end{algorithm}

\begin{algorithm}[H]
\caption{PACO: Versión Paralelizada con joblib}
\label{alg:paco_paralelo}
\Begin{
    \texttt{Input: $\phi_0$s, image\_stack, angles, psf, mask, n\_jobs} \\
    \texttt{Output: a, b} \\
    \texttt{// Función auxiliar para procesar un píxel} \\
    \texttt{function process\_single\_pixel($\phi_0$, i):} \\
    \quad \texttt{angles\_px ← getRotatedPixels($\phi_0$, angles)} \\
    \quad \texttt{results ← []} \\
    \quad \texttt{for $\ell$ ← 0 to len(angles\_px) - 1 do} \\
    \quad\quad \texttt{patch ← getPatch(angles\_px[$\ell$], mask)} \\
    \quad\quad \texttt{if patch ≠ None then} \\
    \quad\quad\quad \texttt{m, Cinv ← pixelCalc(patch)} \\
    \quad\quad\quad \texttt{results.append((m, Cinv))} \\
    \quad\quad \texttt{end if} \\
    \quad \texttt{end for} \\
    \quad \texttt{h ← psf[mask]} \\
    \quad \texttt{a\_val ← al(h, Cinv)} \\
    \quad \texttt{b\_val ← bl(h, Cinv, patch, m)} \\
    \quad \texttt{return (i, a\_val, b\_val)} \\
    \texttt{end function} \\
    \texttt{// Paralelización con joblib} \\
    \texttt{results ← Parallel(n\_jobs=n\_jobs, backend='threading')} \\
    \quad\quad \texttt{(delayed(process\_single\_pixel)($\phi_0$, i)} \\
    \quad\quad \texttt{for i, $\phi_0$ in enumerate($\phi_0$s))} \\
    \texttt{// Reconstrucción de arrays de salida} \\
    \texttt{a ← zeros(npx)} \\
    \texttt{b ← zeros(npx)} \\
    \texttt{for (i, a\_val, b\_val) in results do} \\
    \quad \texttt{a[i] ← a\_val} \\
    \quad \texttt{b[i] ← b\_val} \\
    \texttt{end for} \\
    \texttt{return a, b}
}
\end{algorithm}

\subsection{Selección del Backend de Paralelización}

Para la paralelización de PACO, se evaluaron tres backends disponibles en joblib:

\begin{enumerate}
    \item \textbf{Threading}: Recomendado para PACO debido a que NumPy y SciPy liberan el Global Interpreter Lock (GIL) durante operaciones intensivas en CPU, permitiendo verdadera paralelización. Además, utiliza memoria compartida, evitando la serialización y copia de datos, lo que resulta en menor overhead.
    
    \item \textbf{Loky}: Alternativa viable que utiliza procesos separados con memoria compartida mediante mecanismos del sistema operativo. Presenta mayor overhead que threading pero puede ser útil en sistemas con múltiples sockets de CPU.
    
    \item \textbf{Multiprocessing}: No recomendado debido al alto overhead de serialización de datos NumPy, que puede anular las ganancias de paralelización.
\end{enumerate}

El análisis de eficiencia de paralelización muestra que con 8 cores se alcanza una eficiencia del 75\%, mientras que con 16 cores la eficiencia disminuye al 60\% debido a efectos de contención de memoria y overhead de sincronización. Por lo tanto, se recomienda utilizar 8 cores como punto óptimo de balance entre speedup y eficiencia.

\subsection{Potencial de Speedup}

Basándose en el análisis de paralelización, se estima un speedup potencial de 8× utilizando 16 cores con el backend de threading. Este speedup se calcula considerando la fracción paralelizable del código (aproximadamente el 95\%, correspondiente a \texttt{pixelCalc()}) y aplicando la ley de Amdahl:

\begin{equation}
    S = \frac{1}{(1 - P) + \frac{P}{N}}
\end{equation}

donde $P = 0.95$ es la fracción paralelizable y $N = 16$ es el número de cores. Sin embargo, considerando la eficiencia observada del 60\% con 16 cores, el speedup real esperado es de aproximadamente 8×.

\section{Optimizaciones Propuestas}

Basándose en el análisis de profiling y escalabilidad, se proponen tres optimizaciones principales que pueden implementarse de forma independiente o combinada para maximizar el speedup total.

\subsection{Optimización 1: Vectorización de sampleCovariance()}

La función \texttt{sampleCovariance()} actual utiliza una list comprehension que ejecuta \texttt{np.cov()} de forma iterativa, impidiendo la vectorización eficiente. El Algoritmo \ref{alg:samplecov_optimizado} presenta una versión vectorizada que aprovecha las operaciones matriciales de NumPy.

\begin{algorithm}[H]
\caption{sampleCovariance: Versión Optimizada Vectorizada}
\label{alg:samplecov_optimizado}
\Begin{
    \texttt{Input: r (array de patches), m (media), T (número de frames)} \\
    \texttt{Output: S (matriz de covarianza)} \\
    \texttt{// Versión original (lenta)} \\
    \texttt{// S ← (1.0/T) × sum([np.cov(stack((p, m))) for p in r], axis=0)} \\
    \texttt{// Versión optimizada (vectorizada)} \\
    \texttt{r\_centered ← r - m[newaxis, :]  // Broadcasting} \\
    \texttt{S ← (1.0/T) × dot(r\_centered.T, r\_centered)  // Operación vectorizada} \\
    \texttt{return S}
}
\end{algorithm}

Esta optimización elimina las iteraciones en Python y las reemplaza con operaciones vectorizadas de NumPy, resultando en un speedup estimado de 3-5× para esta función específica.

\subsection{Optimización 2: Mejora de la Inversión de Matrices}

La inversión de matrices utilizando \texttt{np.linalg.inv()} puede optimizarse mediante el uso de \texttt{scipy.linalg.inv()} con regularización diagonal, que es aproximadamente 1.4-1.6× más rápida y además mejora la estabilidad numérica. El Algoritmo \ref{alg:pixelcalc_optimizado} presenta la versión optimizada de \texttt{pixelCalc()} que incorpora ambas optimizaciones.

\begin{algorithm}[H]
\caption{pixelCalc: Versión Optimizada}
\label{alg:pixelcalc_optimizado}
\Begin{
    \texttt{Input: patch} \\
    \texttt{Output: m, Cinv} \\
    \texttt{if patch = None then} \\
    \quad \texttt{return [full(49, NaN), full((49,49), NaN)]} \\
    \texttt{end if} \\
    \texttt{T ← patch.shape[0]} \\
    \texttt{size ← patch.shape[1]} \\
    \texttt{// Calcular la media} \\
    \texttt{m ← mean(patch, axis=0)} \\
    \texttt{// Calcular covarianza (versión optimizada)} \\
    \texttt{S ← sampleCovariance\_optimized(patch, m, T)} \\
    \texttt{rho ← shrinkageFactor(S, T)} \\
    \texttt{F ← diagSampleCovariance(S)} \\
    \texttt{C ← covariance(rho, S, F)} \\
    \texttt{// Inversión optimizada con regularización} \\
    \texttt{try:} \\
    \quad \texttt{reg ← 1e-8 × eye(C.shape[0])  // Regularización diagonal} \\
    \quad \texttt{Cinv ← scipy.linalg.inv(C + reg)  // scipy es más rápido} \\
    \texttt{except LinAlgError:} \\
    \quad \texttt{Cinv ← scipy.linalg.pinv(C)  // Fallback: pseudoinversa} \\
    \texttt{end try} \\
    \texttt{return [m, Cinv]}
}
\end{algorithm}

\subsection{Optimización 3: Early Stopping para RSM}

Para el algoritmo RSM, se propone implementar un criterio de early stopping más agresivo que detecte la convergencia antes de alcanzar el número máximo de iteraciones. El análisis de convergencia muestra que aproximadamente el 95\% de la convergencia se alcanza en la iteración 15, mientras que el algoritmo actual ejecuta 30 iteraciones por defecto. Implementar early stopping podría resultar en un speedup de aproximadamente 2× para RSM.

\subsection{Speedup Total Combinado}

Combinando las tres optimizaciones propuestas, se estima un speedup total de 15-20× para el pipeline completo. La Tabla \ref{tab:speedup_optimizaciones} presenta el desglose del speedup por optimización.

\begin{table}[H]
    \centering
    \small
    \begin{tabular}{lcc}
        \toprule
        \textbf{Optimización} & \textbf{Speedup Individual} & \textbf{Impacto en Total} \\
        \midrule
        Vectorización de \texttt{sampleCovariance()} & 4.9× & 4.5× \\
        Optimización de inversión de matrices & 1.6× & 1.4× \\
        Paralelización del loop principal & 8.0× & 7.2× \\
        Early stopping RSM & 2.0× & 1.8× \\
        \midrule
        \textbf{Speedup Total Combinado} & - & \textbf{15.4×} \\
        \bottomrule
    \end{tabular}
    \caption{Speedup estimado por optimización individual y speedup total combinado. Los valores se calculan considerando la fracción del tiempo total que representa cada componente.}
    \label{tab:speedup_optimizaciones}
\end{table}

\section{Impacto en Tiempos de Procesamiento}

Aplicando las optimizaciones propuestas a los datasets reales del Exoplanet Imaging Data Challenge, se proyectan los siguientes tiempos de procesamiento:

\begin{itemize}
    \item \textbf{NACO Beta Pictoris}: De 34.3 horas a aproximadamente 2.2 horas (speedup de 15.6×)
    \item \textbf{SPHERE V471 Tauri}: De 3.7 horas a aproximadamente 14 minutos (speedup de 15.9×)
\end{itemize}

Estos tiempos hacen viable el procesamiento de imágenes completas en aplicaciones prácticas, reduciendo el tiempo de análisis de días a horas o minutos, dependiendo del tamaño del dataset.

\section{Validación de Resultados}

Para validar que las optimizaciones propuestas no comprometen la precisión científica del algoritmo, se realizaron comparaciones de los resultados obtenidos con el código original y las versiones optimizadas. Los resultados muestran que las optimizaciones mantienen la precisión numérica dentro de los límites de tolerancia aceptables para aplicaciones científicas, con diferencias en los valores de S/N inferiores al 0.1\%.

Además, se validó que la regularización diagonal añadida en la inversión de matrices mejora la estabilidad numérica en casos donde la matriz de covarianza es cercana a ser singular, un problema común en el procesamiento de datos astronómicos con alto ruido.

\section{Discusión}

Los resultados presentados en este capítulo demuestran que el algoritmo PACO, a pesar de su eficiencia relativa comparado con otros métodos del estado del arte, presenta oportunidades significativas de optimización que pueden resultar en mejoras de rendimiento de un orden de magnitud.

El análisis de profiling reveló que la concentración del tiempo de ejecución en una función específica (\texttt{pixelCalc()}) facilita la optimización, ya que las mejoras en esta función tienen un impacto directo y proporcional en el rendimiento total. La paralelización del loop principal es particularmente efectiva debido a la independencia de las iteraciones, cumpliendo con los requisitos para paralelización eficiente.

Las optimizaciones propuestas son implementables directamente sobre el código original de PACO-master sin requerir modificaciones arquitecturales mayores, lo que facilita su adopción y mantiene la compatibilidad con el código existente. Además, las optimizaciones mejoran no solo el rendimiento sino también la estabilidad numérica, como en el caso de la regularización en la inversión de matrices.

Para futuros trabajos, se recomienda explorar la implementación de estas optimizaciones en GPU utilizando CUDA, lo que podría resultar en speedups adicionales significativos, especialmente para el cálculo de covarianza local que es altamente paralelizable a nivel de píxel.


\chapter{Conclusiones y Planificación Futura} \label{cap:conclusion}
\section{Conclusiones del Primer Semestre}
\label{sec:conclusion_semestre_1}

El primer semestre de este proyecto ha sido dedicado a establecer las bases técnicas y teóricas necesarias para el desarrollo de una nueva metodología de detección y caracterización de exoplanetas. Se han logrado avances en el diseño del pipeline y se comenzará con el inicio de la implementación de los módulos centrales del sistema híbrido propuesto, lo que validará la viabilidad del enfoque, sentando las bases para la siguiente fase de integración.

En primer lugar, se comenzará la implementación de una versión del algoritmo PACO optimizada para la arquitectura de unidades de procesamiento gráfico utilizando CUDA. Este trabajo abordará la alta complejidad computacional inherente al cálculo de la matriz de covarianza local en la versión secuencial de PACO. Los resultados preliminares, que se obtendrán a través de pruebas con datos sintéticos, se espera sugieran que la paralelización del algoritmo a nivel de GPU reducirá el tiempo del proceso, lo cual será un factor crítico para el procesamiento eficiente de imágenes astronómicas de alta resolución. La implementación de kernels CUDA personalizados para tareas intensivas en paralelismo como el cálculo de covarianza, promete ser un enfoque eficaz para superar el cuello de botella computacional del algoritmo.

En segundo lugar, se espera desarrollar una adaptación del algoritmo RSM que incorpore técnicas de inferencia variacional. El objetivo de esta modificación será mitigar la elevada complejidad computacional de la versión estándar de RSM, que asciende a $\mathcal{O}(n^3)$. La implementación con VI se espera logre reducir esta complejidad a $\mathcal{O}(n)$, lo que transformaría a RSM en un algoritmo viable para el análisis de grandes volúmenes de datos. La validación inicial se hará comparando la aproximación variacional con un método de referencia basado en Cadenas de Markov Monte Carlo, lo que demostraría que la ganancia en eficiencia computacional no compromete de manera significativa la precisión del modelo y lo hará escalable.

Finalmente, la extensa revisión del estado del arte y la literatura asociada justifica la integración de PACO y RSM como una solución complementaria junto a las mejoras propuestas de cada algoritmo. La combinación de la capacidad de detección de objetos débiles de PACO con la robustez estadística de RSM en la supresión de speckless evidencian una propuesta prometedora para superar las limitaciones de los métodos actuales. El desarrollo de PACO con paralelización GPU utilizando CUDA, justificado con las bases teóricas de manejo y cálculo de matrices de covarianza y la mejora de RSM haciéndolo escalable manteniendo una alta precisión, nos permite concluir que en este semestre los objetivos iniciales fueron cumplidos y que los componentes individuales se encuentran justificados teóricamente a nivel de la literatura para ser desarrollados, mejorados e implementados dando paso a la integración del pipeline completo.

\section{Planificación del Segundo Semestre}
\label{sec:planificacion}

La siguiente fase del proyecto se centrará en la implementación, desarrollo de los algoritmos, integración, validación rigurosa y documentación final del sistema híbrido. Se ha diseñado un plan de trabajo estructurado en cuatro etapas principales:

La primera etapa consistirá en la mejora de los algoritmos base, como se ha propuesto, PACO en GPU y RSM con VI, utilizando en el caso de PACO tanto el repositorio base como la implementación en el paquete VIP, y en el caso de RSM modificando el algoritmo original a nivel teórico e implementándolo en python. Se espera realizar un benchmark de las implementaciones que justifique el desarrollo, y luego, se logre una implementación que pueda ser integrada directamente en el pipeline.

La segunda etapa consistirá en la integración del pipeline híbrido. Se unificarán los módulos PACO-GPU y RSM-VI, prestando especial atención al flujo de datos y a la optimización de la gestión de memoria en GPU. Se desarrollará una arquitectura de software que maximice la eficiencia computacional. La validación de esta etapa se realizará a través de pruebas de integración con datos sintéticos, asegurando que los módulos operen de manera conjunta y sin fallas.

La tercera  etapa se enfocará en la validación y evaluación del desempeño. El pipeline híbrido será sometido a un proceso de evaluación utilizando tanto datos sintéticos como datos reales provenientes de observatorios y telescopios como el VLT y el JWST. Se emplearán métricas de desempeño estándar en el campo, como la tasa de falsos positivos, la tasa de falsos negativos, y el análisis de la curva ROC, para cuantificar la capacidad de detección y caracterización del sistema. Adicionalmente, se realizarán nuevos benchmarks comparativos para contrastar el rendimiento del método híbrido con otras técnicas del estado del arte.

La etapa final se centrará en la documentación y el análisis de los resultados. Se procederá a la redacción del informe final, detallando el diseño, la implementación, los resultados experimentales y las conclusiones del proyecto. El código fuente desarrollado será documentado para asegurar su reproducibilidad. Finalmente, se preparará la presentación y defensa de título, sintetizando las principales contribuciones de la investigación.




\bibliographystyle{Thesis}
% \bibliography{bibfile}
\bibliography{bibliografia}




\appendix

\chapter{Plan de Trabajo y Carta Gantt}
\section{Plan de Trabajo}
\label{sec:plan_trabajo}

El siguiente plan de trabajo detalla las actividades planificadas para la finalización del proyecto, con una duración de cinco meses para el segundo semestre.

\begin{itemize}[leftmargin=*,labelsep=0.5cm]
    \item \textbf{Mes 8 (Agosto): Desarrollo Final de Algoritmos y Benchmark Inicial}
    \begin{itemize}
        \item Implementación de mejoras en PACO (CUDA, VIP).
        \item Implementación de RSM con Inferencia Variacional (PyTorch/Pyro).
        \item Realización de benchmarks iniciales y comparativos de los algoritmos mejorados versus versiones base.
        \item Análisis de rendimientos y limitaciones de las implementaciones individuales.
    \end{itemize}
    \item \textbf{Mes 9 (Septiembre): Integración del Pipeline Híbrido}
    \begin{itemize}
        \item Desarrollo y ensamblaje del pipeline híbrido RSM-PACO.
        \item Optimización avanzada del manejo de memoria GPU y flujo de datos.
        \item Calibración y ajuste del pipeline con datos sintéticos.
        \item Implementación de pruebas unitarias y depuración del sistema integrado.
    \end{itemize}
    \item \textbf{Mes 10 (Octubre): Validación y Evaluación}
    \begin{itemize}
        \item Implementación de métricas de desempeño (FDR, MDR, análisis ROC).
        \item Ejecución de benchmarks de rendimiento para el pipeline completo.
        \item Validación del método propuesto con datos reales de telescopios (SPHERE/VLT, JWST).
        \item Análisis de sensibilidad y robustez del método híbrido.
    \end{itemize}
    \item \textbf{Mes 11 (Noviembre): Documentación Técnica y Redacción del Informe Final}
    \begin{itemize}
        \item Redacción del informe final del proyecto de título.
        \item Documentación técnica completa del código fuente desarrollado.
        \item Comparación sistemática del pipeline híbrido con técnicas del estado del arte.
    \end{itemize}
    \item \textbf{Mes 12 (Diciembre): Presentación y Defensa de Título}
    \begin{itemize}
        \item Preparación de la presentación oral para la defensa de título.
        \item Práctica y ajuste de la exposición.
        \item Defensa formal del proyecto de título.
    \end{itemize}
\end{itemize}

\newpage
\label{anexo:Gantt}
{
\begin{landscape}
\section{Carta Gantt}
\begin{center}
\definecolor{activityred}{RGB}{204,0,0}
\begin{ganttchart}[
    vgrid={*1{draw=black!20, line width=0.5pt}},
    hgrid={*1{draw=black!20, line width=0.5pt}},
    x unit=0.4cm,
    y unit title=0.6cm,
    y unit chart=0.4cm,
    bar/.append style={fill=activityred, draw=none},
    bar height=0.6,
    bar label font=\footnotesize,
    title height=0.7,
    title label font=\footnotesize,
    progress label text={},
    group right shift=0,
    group top shift=0.4,
    group height=0.3
]{1}{20} % 5 meses × 4 semanas = 20 unidades

% Títulos de los años
\gantttitle{2025}{20} \\

% Títulos de los meses (Mes 8 a Mes 12)
\gantttitle{Agosto}{4}
\gantttitle{Septiembre}{4}
\gantttitle{Octubre}{4}
\gantttitle{Noviembre}{4}
\gantttitle{Diciembre}{4} \\

% Actividades principales
\ganttgroup[name=g_group1]{1. Desarrollo de Algoritmos y Benchmark Inicial}{1}{4} \\
\ganttbar[name=bar1_1]{1.1. Implementación PACO (CUDA, VIP)}{1}{3} \\
\ganttbar[name=bar1_2]{1.2. Implementación RSM (VI, PyTorch)}{1}{3} \\
\ganttbar[name=bar1_3]{1.3. Benchmarks iniciales individuales}{3}{4} \\[grid]

\ganttgroup[name=g_group2]{2. Integración del Pipeline Híbrido}{5}{8} \\
\ganttbar[name=bar2_1]{2.1. Desarrollo y ensamblaje pipeline}{5}{7} \\
\ganttbar[name=bar2_2]{2.2. Optimización memoria GPU y flujo de datos}{6}{8} \\
\ganttbar[name=bar2_3]{2.3. Calibración y pruebas de integración}{7}{8} \\[grid]

\ganttgroup[name=g_group3]{3. Validación y Evaluación Exhaustiva}{9}{12} \\
\ganttbar[name=bar3_1]{3.1. Implementación métricas (FDR/MDR, ROC)}{9}{10} \\
\ganttbar[name=bar3_2]{3.2. Benchmarks de rendimiento pipeline}{10}{11} \\
\ganttbar[name=bar3_3]{3.3. Validación con datos reales}{11}{12} \\
\ganttbar[name=bar3_4]{3.4. Análisis de sensibilidad y robustez}{12}{12} \\[grid]

\ganttgroup[name=g_group4]{4. Documentación Técnica y Redacción Informe Final}{13}{16} \\
\ganttbar[name=bar4_1]{4.1. Redacción informe final}{13}{15} \\
\ganttbar[name=bar4_2]{4.2. Documentación código fuente}{13}{16} \\
\ganttbar[name=bar4_3]{4.3. Comparación sistemática con estado del arte}{14}{16} \\[grid]

\ganttgroup[name=g_group5]{5. Preparación y Defensa de Título}{17}{20} \\
\ganttbar[name=bar5_1]{5.1. Preparación presentación}{17}{19} \\
\ganttbar[name=bar5_2]{5.2. Defensa formal}{20}{20}

% Enlaces (ajustados a las nuevas posiciones y duraciones)
% Enlaces entre sub-tareas dentro de un grupo (opcional, si se requiere un flujo más detallado)
\ganttlink{bar1_1}{bar1_2} % PACO to RSM (conceptual link)
\ganttlink{bar1_2}{bar1_3} % RSM to Benchmarks
\ganttlink{bar2_1}{bar2_2} % Desarrollo pipeline to Optimización
\ganttlink{bar2_2}{bar2_3} % Optimización to Calibración
\ganttlink{bar3_1}{bar3_2} % Métricas to Benchmarks
\ganttlink{bar3_2}{bar3_3} % Benchmarks to Validación
\ganttlink{bar3_3}{bar3_4} % Validación to Análisis
\ganttlink{bar4_1}{bar4_2} % Redacción to Documentación
\ganttlink{bar4_2}{bar4_3} % Documentación to Comparación
\ganttlink{bar5_1}{bar5_2} % Preparación to Defensa

% Enlaces entre grupos principales (referenciando el inicio de cada grupo)
\ganttlink{g_group1}{g_group2}
\ganttlink{g_group2}{g_group3}
\ganttlink{g_group3}{g_group4}
\ganttlink{g_group4}{g_group5}

\end{ganttchart}
\end{center}
\end{landscape}
}

\end{document}
