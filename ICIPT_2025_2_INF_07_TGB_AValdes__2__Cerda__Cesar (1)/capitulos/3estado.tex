%* estado del arte es la etapa del proceso de investigación posterior a la elección y justificación del tema. Una vez que hayas escogido su tema de interés, sigue la parte en la que se tendrá que leer lo que se ha escrito en torno a esa temática.\\
%* Estado del arte es: Estudio analítico del conocimiento acumulado que hace parte de la investigación documental y que tiene como objetivo inventariar y sistematizar la producción en un área del conocimiento de los últimos años”
%Introducción al capítulo

El estado del arte constituye una revisión sistemática del conocimiento acumulado en el campo de la detección de exoplanetas mediante técnicas de aprendizaje estadístico y procesamiento de imágenes de alto contraste. Este capítulo sintetiza los avances recientes en:

\begin{enumerate}
    \item Métodos Tradicionales de Sustracción de PSF y su Evolución
    \item Avances en Modelado de Ruido y Detección Estadística
    \item Brechas tecnológicas que justifican el desarrollo de un enfoque híbrido RSM-PACO para observación directa.
\end{enumerate}


\section{Métodos tradicionales} Se abordan técnicas como PCA, sustracción PSF y LOCI para dar contexto de cuáles se utilizan y cómo deben compararse a la solución propuesta.
\subsection{PCA} 
Es una técnica de reducción de dimensionalidad ampliamente adoptada en la imagen de alto contraste para la supresión de speckles. En esencia, PCA transforma datos de alta dimensión en un espacio de menor dimensión al identificar direcciones ortogonales (componentes principales) que capturan la máxima varianza en el conjunto de datos. En el contexto de las imágenes astronómicas, esto implica realizar una Descomposición de Valores Singulares (SVD) en el cubo de imágenes para identificar los patrones de ruido dominantes, que luego se restan. \\

A pesar de su eficiencia computacional y uso generalizado, una limitación significativa de PCA en la detección de exoplanetas es su propensión a la ``auto-sustracción''. Esto ocurre porque PCA puede aprender inadvertidamente la forma de la PSF del telescopio, que, para un alto número de componentes de PCA, captura no solo el ruido estructurado de los speckles sino también la débil y característica forma de la propia señal del planeta. Esto conduce a una pérdida de sensibilidad, particularmente en regiones cercanas a la estrella (por ejemplo, por debajo de 0.3 arcosegundos), donde la señal del planeta es más débil y más susceptible a la sobre-sustracción. En consecuencia, la dependencia de PCA del número de componentes a menudo requiere un ajuste manual, lo que complica la optimización de hiperparámetros. La limitación inherente de PCA de la ``auto-sustracción'' revela una compensación fundamental y persistente en la supresión de speckles: cuanto más agresivamente un algoritmo intenta modelar y eliminar el abrumador ruido estelar, mayor es el riesgo de atenuar o eliminar simultáneamente la débil y deseada señal astrofísica. Esta situación subraya la necesidad crítica de modelos de ruido más sofisticados que puedan distinguir robustamente entre el ruido y las señales verdaderas, en lugar de simplemente identificar patrones dominantes. Esto impulsa el desarrollo de métodos que incorporan priors físicos o una discriminación estadística más matizada, buscando superar esta dicotomía intrínseca entre la supresión de ruido y la preservación de la señal.\\

\subsection{Combinación de Imágenes Optimizada Localmente (LOCI)}
El algoritmo LOCI, introducido por \citep{Lafreniere_2007} en 2007, representa un avance en las técnicas de sustracción de PSF. LOCI opera construyendo una imagen de PSF de referencia optimizada a partir de un conjunto de imágenes de referencia disponibles. Esta imagen de referencia se construye como una combinación lineal de imágenes seleccionadas, donde los coeficientes de la combinación se optimizan de forma independiente dentro de múltiples subsecciones localizadas de la imagen para minimizar el ruido residual. Esta estrategia de optimización local permite a LOCI adaptarse a la naturaleza espacialmente variable del ruido de los speckles.  \\

Aunque es altamente eficaz en la supresión de speckles y en el logro de un alto contraste, los métodos basados en LOCI son conocidos por introducir sesgos en la astrometría (posición) y la fotometría (brillo) de las fuentes débiles detectadas. Esto se debe a menudo a la inevitable auto-sustracción parcial de la señal del planeta durante el proceso de modelado del ruido, lo que puede distorsionar las verdaderas características del exoplaneta. \\

La fortaleza de LOCI radica en su optimización local, que es muy adecuada para manejar la naturaleza no estacionaria del ruido de los speckles. Sin embargo, este enfoque localizado, cuando se implementa sin suficiente contexto global o restricciones físicas, puede conducir inadvertidamente a inconsistencias en toda la imagen o a sesgos en los parámetros astrofísicos derivados. Esto pone de manifiesto un desafío crítico en el procesamiento de imágenes: equilibrar la necesidad de adaptabilidad local al ruido complejo con el requisito de una recuperación de la señal globalmente consistente y físicamente precisa. Si las optimizaciones locales son demasiado agresivas o no están limitadas por información global (por ejemplo, la forma esperada de una PSF o la variación suave del fondo a través del campo), pueden introducir artefactos o distorsionar las señales de maneras que son localmente óptimas para la sustracción de ruido, pero perjudiciales para la interpretación científica general.\newpage
\section{ Avances en Modelado de Ruido y Detección Estadística} Se abordan los algoritmos/métodos más utilizados actualmente.
\subsection{PACO}
PACO es un algoritmo híbrido que se sitúa en el núcleo de los pipelines de reducción de ruido, combinando la descomposición matricial con el modelado estadístico espacial.\\

PACO se distingue por emplear un enfoque de aprendizaje estadístico para modelar las fluctuaciones de fondo, incorporando covarianzas de parches no estacionarios, lo que le permite aprender un modelo gaussiano multivariante no estacionario del entorno. Esto es crucial para una estimación precisa del fondo y lo diferencia de muchos métodos anteriores que se basaban en restas o transformaciones más sencillas. PACO es fotométricamente insesgado y permite evaluar directamente la tasa de falsas alarmas sin simulaciones. La implementación de este proceso dual en CUDA, utilizando cuBLAS para la SVD y kernels personalizados para la covarianza, busca una aceleración de 50x respecto a las versiones de CPU. \\

A pesar de sus ventajas, PACO presenta limitaciones. Aunque es eficiente y puede ejecutarse en GPU, sufre una pérdida de sensibilidad en zonas cercanas a la estrella  debido a un suavizado excesivo que elimina tanto el ruido como las señales débiles. Además, requiere un ajuste manual del número de componentes de PCA, lo que complica la optimización de hiperparámetros. La investigación sobre PACO ha continuado, con trabajos como \citep{Flasseur_2020a} (2020), que describe PACO ASDI, un algoritmo para la detección y caracterización de exoplanetas en imagen directa con espectrógrafos de campo integral, y \citep{Flasseur_2020b} (2020), que aborda la robustez frente a fotogramas defectuosos. El trabajo original de PACO fue publicado por \citep{Flasseur_2018} en 2018.



\subsection{RSM}
El Modelo de Cambio de Régimen (RSM) es una solución para el problema de clasificación píxel a píxel en condiciones de incertidumbre extrema (SNR < 3). Este modelo probabilístico jerárquico considera que cada píxel puede estar en dos regímenes: ruido o planeta, con probabilidades $\pi$ y 1-$\pi$ respectivamente. El RSM modela explícitamente la probabilidad de que un píxel dado contenga solo ruido residual o una señal planetaria genuina, ofreciendo ventajas teóricas como la cuantificación natural de la incertidumbre y una mayor robustez frente a falsos positivos. Sin embargo, es computacionalmente demasiado complejo debido al usom de MCMC. Para superar esto, se propone el uso de Inferencia Variacional, que aproxima la posterior mediante distribuciones gaussianas factorizadas optimizadas con gradiente descendente. Esta aproximación reduce la complejidad a O(n), permitiendo procesar imágenes completas en minutos en lugar de días. Un aspecto clave es la inicialización inteligente de los parámetros basada en estadísticas locales de los residuos de PACO, lo que acelera la convergencia del algoritmo. El trabajo seminal que introdujo el RSM para la detección de exoplanetas fue realizado por \citep{Dahlqvist_2020} en 2020, con mejoras posteriores en \citep{Dahlqvist_2021} en 2021.

\subsection{Otros Métodos Recientes, Paquetes de Software para Procesamiento de Imágenes de Alto Contraste y Brechas Actuales}

VIP (Vortex Image Processing): VIP es un paquete de Python de código abierto dedicado a la imagen de alto contraste astronómica, desarrollado para proporcionar un marco flexible para el procesamiento de datos e imágenes \citep{Gomez_Gonzalez_2017}. Fue iniciado por Carlos Alberto Gomez Gonzalez y ha sido mantenido y desarrollado por el Dr. Valentin Christiaens desde 2020. VIP implementa funcionalidades para construir pipelines de procesamiento de datos de alto contraste, abarcando algoritmos de pre- y post-procesamiento, estimación de posición y flujo de fuentes potenciales, y generación de curvas de sensibilidad \citep{Gomez_Gonzalez_2017}. Entre las técnicas de sustracción de PSF de referencia para el post-procesamiento ADI, VIP incluye varias variantes de algoritmos basados en PCA (como PCA anular e incremental) y una implementación de NMF \citep{Gomez_Gonzalez_2017}. VIP también ofrece la posibilidad de realizar SVD en GPU utilizando CuPy o PyTorch, y optimizar rutinas de corrección de píxeles defectuosos con Numba, lo que puede ofrecer mejoras de velocidad significativas (hasta 50x para rotaciones de imagen con OpenCV) \\


PynPoint, un paquete de código abierto en Python, ofrece un enfoque modular basado en PCA y análisis estadístico independiente, optimizado para datos de SPHERE/VLT y JWST. Sus ventajas incluyen la integración modular de múltiples algoritmos (PCA, NMF, sustracción de PSF) y precisión fotométrica con correcciones de sesgo y estimación de incertidumbres. Sin embargo, PynPoint no modela el ruido no estacionario como PACO ni segmenta regímenes como RSM, y su dependencia de PCA limita la sensibilidad en contrastes bajos (<1e-6). El trabajo principal de PynPoint fue publicado por \citep{Stolker_2019} en 2019.La Factorización de Matrices No Negativas (NMF) es otra técnica iterativa aplicada al postprocesamiento de datos de imagen directa para la supresión de speckles. NMF construye una base de componentes no ortogonales y no negativos a partir de imágenes de referencia. Sus ventajas incluyen no requerir una selección previa de referencias y una mejor preservación de la morfología de discos débiles. Aunque computacionalmente más costosa que KLIP para la construcción de componentes, NMF es menos propensa al sobreajuste y mejora la eliminación de ruido. El trabajo de \citep{Ren_2018} en 2018 es un ejemplo clave de la aplicación de NMF en este campo.\\

Los enfoques de aprendizaje profundo (Deep Learning) han emergido como una vía prometedora para mejorar la detección de exoplanetas y la reducción de speckles, a menudo combinando modelos estadísticos con redes neuronales. Ejemplos incluyen deep PACO \citep{Flasseur_2023} (2023), que combina el modelo estadístico de PACO con redes neuronales convolucionales para mejorar la sensibilidad de detección, y torchKLIP \citep{Soummer_2012} (2012), un paquete de PyTorch que busca optimizar la sustracción de PSF. PACOME, una extensión de PACO, se enfoca en la combinación óptima de observaciones de múltiples épocas para la detección conjunta de exoplanetas y la estimación de órbitas. Este algoritmo integra la exploración de órbitas keplerianas dentro de un formalismo de detección estadística, permitiendo la co-adición constructiva de señales débiles y mejorando la sensibilidad de detección con el número de épocas. El trabajo clave para PACOME fue publicado por \citep{Dallant_2023} en 2023.\\

A pesar de estos avances, existe una brecha técnica importante: la necesidad de un marco que preprocese eficientemente (acelerando PACO en GPU) y clasifique señales con modelos bayesianos escalables (como RSM que utiliza inferencia variacional). La dependencia de PCA en PynPoint, por ejemplo, limita la sensibilidad en bajos contrastes, justificando la integración de componentes como PACO (para ruido no estacionario) y RSM (para segmentación temporal). Esto subraya la necesidad de soluciones híbridas que combinen lo mejor de diferentes enfoques para abordar los desafíos persistentes en la detección de exoplanetas de alto contraste. 

